\newpage
\section{Semantic segmentation}

This section presents the most successful methods involving neural networks and supervised learning. In semantic segmentation, one assigns a class to each pixel of an input image, unlike in the classification task, where one classifies the entire image. 

Segmentation has always been one of the most fundamental areas of computer vision. The classic approaches are mostly based on the standard signal processing theory and some of them can still be implemented and give satisfactory results. However, this applies only to a limited number of use cases, where the conditions are very close to ideal and where the robustness of the algorithm is not crucial. To give an example of classic methods, one can refer to thresholding, region growing and mean-shift segmentation \cite{coufal}. More advanced methods which use machine learning classification have also been introduced, such as TextonBoost, TextonForest and Random Forest \cite{segnet} \cite{bayesian}. These algorithms have fallen out of favour due to the massive success of ANN.

\subsection{Encoder-decoder architecture}

In the previous chapter, CNN architectures designed for image classification were presented. The size of the output layer of these networks is determined by the number of categories of classification because the CNN transfers to an FCN in the end. In semantic segmentation, however, one needs to get an image of the same resolution as the input image containing the information about a class of every pixel. To do this, the common scheme is introduced: the first part of the network is left unchanged but now, instead of the transition to FCN, various methods are implemented to upsample the encoded image features from the deepest layer of the CNN. This scheme is referred to as the encoder-decoder architecture. \cite{stanford-L11}

\vspace{4mm}
\begin{figure}[h]
	\begin{center}
		\includegraphics*[width=11cm, keepaspectratio]{obr/segnet.png}
	\end{center}
	\vspace{4mm}
	\caption{SegNet - an example of encoder-decoder CNN architecture. \cite{segnet}} 
	\label{encoder}
\end{figure}

The purpose of the encoder is to downsample the input images while still representing their significant features. The decoder part of the algorithm then upsamples the output of the encoder to the original input image size. This is usually done by performing reverse operations to max-pooling and convolution. The last part of the decoder typically gives the final segmented image. \\

Shortly after the success of CNN in image classification challenges, there have been several segmentation architectures introduced which use CNN as the encoder. Some of the state-of-the-art architectures were, for instance, FCN, DeconvNet and U-Net. These networks share the idea of having CNN incorporated as the encoder but differ in the form of the decoder part. However, the problem of training such networks due to a large number of trainable parameters, the design of the decoder and hence the need of introducing the cumbersome multi-stage training made them very difficult to use in practice. SegNet \cite{segnet}, introduced in 2015, differs from these architectures as it has a significantly lower number of parameters and the design of the encoder-decoder network allows it to be trained via standard method using backpropagation and SGD.  \cite{segnet}

\subsubsection{Input upsampling}

The upsampling in the decoding part of the network is done via two mechanisms: learnable transposed convolution and unpooling. 

Transposed convolution, just like the standard convolution used in CNNs, uses learnable filters. The difference is that it takes a single input point instead of a region, uses it to multiply each element of the filter and creates its imprint in the output layer. This scheme is illustrated in Figure \ref{transposed} (left). \cite{stanford-L11}

\vspace{4mm}
\begin{figure}[h]
	\begin{center}
		\includegraphics*[width=5cm, keepaspectratio]{obr/transposed.png}
	\end{center}
	\vspace{4mm}
	\caption{Transposed convolution. \cite{theano}} 
	\label{transposed}
\end{figure}

There are several ways to impelement unpooling. In an encoder-decoder architecture, the corresponding layers in the encoder and decoder can, for example, share the original locations of the elements that were pooled in the encoding part. The decoder then uses these indices for upsampling, as shown in Figure \ref{transposed} (right). This reconstructs the original positions of the features in the original image. Unpooling operation does not have any learnable parameters. \cite{segnet} \cite{stanford-L11}

\vspace{4mm}
\begin{figure}[h]
	\begin{center}
		\includegraphics*[width=11cm, keepaspectratio]{obr/unpool.png}
	\end{center}
	\vspace{4mm}
	\caption{Max-unpooling. The locations of the maximum elements were saved during max-pooling. The remaining elements are set to zero.} 
	\label{unpool}
\end{figure}

\newpage
\subsection{SegNet}

SegNet is a deep encoder-decoder architecture for multi-class semantic segmentation researched and developed by members of the Computer Vision and Robotics Group at the University of Cambridge. \cite{segnet_tut}

The architecture consists of a sequence of encoders and a corresponding set of decoders followed by a pixel-wise Softmax classifier. Typically, each encoder consists of one or more convolutional layers with batch normalisation and a ReLU non-linearity, followed by max-pooling. SegNet uses max-pooling indices in the decoders to perform upsampling of low-resolution activation maps (Figure \ref{encoder}). The entire architecture can be trained using stochastic gradient descent. \cite{segnet_tut}

\subsubsection{SegNet - encoder}

The architecture of the encoder network is topologically identical to the 13 convolutional layers in the VGG16 network. Each encoder in the encoder network performs convolution with a filter bank to produce a set of activation maps. These are then batch normalized. Then an element-wise ReLU is applied. Following that, max-pooling (with a 2×2 window and stride 2) is performed. Storing the max-pooling indices, i.e, the locations of the maximum feature value in each pooling window is memorized for each encoder feature map. \cite{segnet}

\subsubsection{SegNet - decoder}

The decoders in the network upsample their input feature maps using the memorized max-pooling indices from the corresponding encoder feature maps. These feature maps are then convolved (using transposed convolution) with a trainable decoder filter bank to produce dense feature maps. A batch normalization step is then applied to each of these maps. The high dimensional feature representation at the output of the final decoder is fed to a trainable soft-max classifier. The predicted segmentation corresponds to the class with maximum probability at each pixel. \cite{segnet} The schematic of the SegNet architecture can be found in Attachment XY.

\subsection{Bayesian SegNet}

Bayesian SegNet is a probabilistic variant of SegNet. It can predict pixel-wise class labels together with a measure of model uncertainty.  This is achieved by Monte Carlo sampling with dropout at test time. The authors of the paper show that modelling uncertainty improves segmentation performance by 2-3 \% compared to SegNet. The schematic of the Bayesian SegNet architecture can be found in Attachment XY. \cite{bayesian}

\subsubsection{Monte Carlo Dropout}

Monte Carlo Dropout (MCDO) sampling helps us understand the model uncertainty of the result. As explained in Chapter \ref{dropout_sec}, the standard weight averaging dropout proposes to remove dropout at test time and scale the weights proportionally to the dropout percentage. MCDO, on the other hand, samples the network with randomly dropped out units at test time. \cite{bayesian}

It is important to highlight that the probability distribution from MCDO sampling is significantly different from the ‘probabilities’ obtained from a softmax classifier. The softmax function approximates relative probabilities between the class labels, but not an overall measure of the model’s uncertainty. \cite{bayesian}

\subsection{Evaluating segmentation performance}

The performance of semantic segmentation is often described by so called IoU (intercestion over union) metrics. IoU is the area of overlap between the predicted segmentation and the ground truth divided by the area of union between the predicted segmentation and the ground truth, as shown in the figure below. This metric ranges from 0–1 (0–100\%) with 0 signifying no overlap and 1 signifying perfectly overlapping segmentation. \cite{iou}

\vspace{4mm}
\begin{figure}[h]
	\begin{center}
		\includegraphics*[width=9cm, keepaspectratio]{obr/iou.png}
	\end{center}
	\vspace{4mm}
	\caption{Intersection over union. \cite{iou}} 
	\label{iou}
\end{figure}




