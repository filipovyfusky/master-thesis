\fakulta{Faculty of Mechanical Engineering}
\enfakulta{Fakulta strojního inženýrství}
\adresafakulta{Technická 2896/2, 61669 Brno}

\ustav{Institute of Solid Mechanics, Mechatronics and Biomechanics}
\enustav{Ústav mechaniky těles, mechatroniky a biomechaniky}

% udaje o autorovi

\autor{Bc.}{Filip Špila}{}  % Jméno autora, 
% Tituly vložte samostatně, např. \autor{Ing.}{Petra Smékalová}{}
\autorzkr{Špila, F.}
% bibliografické jméno

\typstudia{N}
% M, N, B, D
% M - Magisterské, N - Navazující magisterské, B - Bakalářské, D-Doktorské
% U typu studia M a N se liší anglický název

\nazev{Semantic segmentation of images using convolutional neural networks} 
% Ručně můžete dlouhý text zalomit pomocí " \break "
\ennazev{Sémantická segmentace obrazu pomocí konvolučních neuronových sítí} 
% Ručně můžete dlouhý text zalomit pomocí " \break "

%vedouci prace
\vedouci{doc. Ing.}{Jiří Krejsa}{, Ph.D.}
\citacevedouci{Vedoucí  doc. Ing. Jiří Krejsa, Ph.D.} % Označení vedoucího práce pro citaci záv. práce. Musí být ukončeno tečkou.

\datumobhajoby{neuvedeno}
\enabstrakt{This thesis deals with the research and implementation of selected architectures of convolutional neural networks (CNNs) for image segmentation. The fundamental terms from the theory of neural networks are summarized in the first part. It also presents the power of CNNs in the field of image data classification. The theoretical part concludes with the research focused on the particular network architecture and its variants used for scene segmentation. In the practical part, the Caffe implementation of the network is taken from its authors and tailored to the specific needs of this study. The steps required to properly set up the software and hardware environments are an essential part of the process. Therefore, the corresponding chapter gives a step-by-step guide that is especially helpful to new Linux users. A custom dataset containing 2600 segmented images is created and used for training all variants of the selected network. Several adjustments of the original implementation are performed, especially for applying the method of using pre-trained parameters of the networks. The training phase includes a selection of hyperparameters, such as the type of optimization algorithm. Finally, the performance and computational cost of the variants of the trained network are evaluated on a testing dataset.} 
% Před "\n" vložit další "\n"
\abstrakt{Tato práce se zabývá rešerší a implementací vybraných architektur konvolučních neuronových sítí pro segmentaci obrazu. V první části jsou shrnuty základní pojmy z teorie neuronových sítí. Tato část také představuje silné stránky konvolučních sítí v oblasti rozpoznávání obrazových dat. Teoretická část je uzavřena rešerší zaměřenou na konkrétní architekturu používanou na segmentaci scén. Implementace této architektury a jejích variant v Caffe je převzata a upravena pro konkrétní použití v praktické části práce. Nedílnou součástí tohoto procesu jsou kroky potřebné ke správnému nastavení softwarového a hardwarového prostředí. Příslušná kapitola proto poskytuje přesný návod, který ocení zejména noví uživatelé Linuxu. Pro trénování všech variant vybrané sítě je vytvořen vlastní dataset obsahující 2600 obrázků. Je také provedeno několik nastavení původní implementace, zvláště pro účely použití předtrénovaných parametrů. Trénování zahrnuje ladění hyperparametrů, jakými jsou například typ optimalizačního algoritmu a rychlost učení. Na závěr je provedeno vyhodnocení výkonu a výpočtové náročnosti všech natrénovaných sítí na testovacím datasetu.}  % Před "\n" vložit další "\n"

\enklicovaslova{semantic segmentation, convolutional neural networks, SegNet, Caffe, Ubuntu} % Před "\n" vložit další "\n"

\klicovaslova{sémantická segmentace, konvoluční neuronové sítě, SegNet, Caffe, Ubuntu} % Před "\n" vložit další "\n"

  
%%
%%   Konec generování údajů
%%


%%
%%   Vlastní vysázení desek umístnit na začátek práce
%%
%\titul% vytiskne titul práce
%\abstrakty% vytiskne stránku s abstrakty
