\fakulta{Faculty of Mechanical Engineering}
\enfakulta{Fakulta strojního inženýrství}
\adresafakulta{Technická 2896/2, 61669 Brno}

\ustav{Institute of Solid Mechanics, Mechatronics and Biomechanics}
\enustav{Ústav mechaniky těles, mechatroniky a biomechaniky}

% udaje o autorovi

\autor{Bc.}{Filip Špila}{}  % Jméno autora, 
% Tituly vložte samostatně, např. \autor{Ing.}{Petra Smékalová}{}
\autorzkr{Špila, F.}
% bibliografické jméno

\typstudia{N}
% M, N, B, D
% M - Magisterské, N - Navazující magisterské, B - Bakalářské, D-Doktorské
% U typu studia M a N se liší anglický název

\nazev{Semantic segmentation of images using convolutional neural networks} 
% Ručně můžete dlouhý text zalomit pomocí " \break "
\ennazev{Sémantická segmentace obrazu pomocí konvolučních neuronových sítí} 
% Ručně můžete dlouhý text zalomit pomocí " \break "

%vedouci prace
\vedouci{doc. Ing.}{Jiří Krejsa}{, Ph.D.}
\citacevedouci{Vedoucí  doc. Ing. Jiří Krejsa, Ph.D.} % Označení vedoucího práce pro citaci záv. práce. Musí být ukončeno tečkou.

\datumobhajoby{neuvedeno}
\abstrakt{} 
  % Před "\n" vložit další "\n"
\enabstrakt{} 
  % Před "\n" vložit další "\n"
\klicovaslova{} % Před "\n" vložit další "\n"
\enklicovaslova{} % Před "\n" vložit další "\n"
  
%%
%%   Konec generování údajů
%%


%%
%%   Vlastní vysázení desek umístnit na začátek práce
%%
%\titul% vytiskne titul práce
%\abstrakty% vytiskne stránku s abstrakty
