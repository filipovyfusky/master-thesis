\chapter{Introduction}
Image segmentation is one of the essential parts of computer vision and autonomous systems alongside with object detection and object recognition. The goal of semantic segmentation is to automatically assign a label to each object of interest (person, animal, car, etc.) in a given image while drawing the exact boundary of it and to do this as robustly and reliably as possible. 

We can see a real-world example in Figure \ref{segment}. Each pixel of the image has been assigned to a specific label and represented by a different colour: red for people, blue for cars, green for trees, etc. This is unlike the image classification task where we classify the image scene as a whole. It is important to say that semantic segmentation is different from so-called instance segmentation where one not only cares about drawing boundaries of objects of a certain class but also wants to distinguish between different instances of the given class \cite{stanford-L11}. For instance, all people in Figure \ref{segment} (each instance of the 'person' class) would have a different colour.

Semantic segmentation has many different applications in fields such as driving autonomous vehicles, human-computer interaction, robotics and various software tools. The most recent developments show increasing demand for reliable object recognition in self-driving vehicles because the driving models must understand the context of the environment they are operating in. \cite{mwiti}

The presented work focuses on research and implementation of one particular segmentation method that uses convolutional neural networks (CNNs). CNNs belong to the family of machine learning algorithms and received attention mainly due to their success in image classification challenges (ImageNet). They subsequently found their use in segmentation tasks where researchers take the most well-performing CNN architectures and use them as the first stage of the algorithm.

\vspace{5mm}
\begin{figure}[htb]
	\begin{center}
		\includegraphics*[width=13cm, keepaspectratio]{obr/semseg.jpg}
	\end{center}
	\caption{Segmentation of an urban road scene \cite{sergios}} 
	\label{segment}
\end{figure}

\chapter{Problem Statements}
The goal of this thesis consists of several points. Firstly, a promising segmentation method using CNNs needs to be found and implemented. It is expected that the neural network will be as straightforward as possible while being capable of giving satisfactory results for the chosen use case (segmentation of a path in an outdoor environment for robot navigation). The images used to train and validate the performance of the network will be provided by the supervisor of the thesis. Also, the author will pick an appropriate software tool for creating Ground Truths\footnote{Manually created image-labels that serve as a reference for the network so that it validates its current accuracy of prediction and computes the needed adjustments of its parameters to get closer to the desired output} and use them to create the training and validating datasets. Lastly, the network should be trained with various sets of training parameters to get a better idea of the network's behaviour and to ensure the best possible results. 

