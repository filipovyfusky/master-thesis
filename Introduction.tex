\chapter{Introduction}
Image segmentation is one of the essential parts of computer vision and autonomous systems alongside with object detection and object recognition. The goal of sematic segmenation is to automatically assign a label to each object of interest (person, animal, car etc.) in a given image while drawing the exact boundary of it and to do this in the most robust and reliable way possible. Speaking in terms of machine learning, each pixel of the input image is intended to belong to a specific class.

We can see a real-world example in Figure 1. Each pixel of the image has been assigned to a specific label and represented by a different color. Red for people, blue for cars, green for trees etc. This is unlike mere image classification task where we classify the image scene as a whole. It is appropriate to say that semantic segmentation is different from so called instance segmentation, where one not only cares about drawing boundaries of objects of a certain class but also wants to distinguish between different instances of the given class. For instance, all people in the image (each instance of the 'person' class) would all have a different colour.

It turns out that semantic segmentation has many different applications in the fields such as driving autonomous vehicles, human-computer interaction, robotics, and photo editing/creativity tools. The most recent development shows the increasing need for reliable object recognition in self-driving cars because it is crucial for the models to understand the context of the environment they’re operating in. %[https://heartbeat.fritz.ai/a-2019-guide-to-semantic-segmentation-ca8242f5a7fc]

The presented work focuses on research and implementation of one particular segmentation method that uses convolutional neural networks (CNNs). CNNs belong to the family of machine learning algorithms and got under attention mainly due to their groundbreaking success in image classification challenges (ImageNet etc,). They subsequently found their use in segmentation tasks where researchers take the most well-performing CNN architectures and use it as the first stage of the segmentation algorithm.

\vspace{5mm}
\begin{figure}[htb]
	\begin{center}
		\includegraphics*[width=13cm, keepaspectratio]{obr/semseg.jpg}
	\end{center}
	\caption{Segmentation of an urban road scene} %[https://theaisummer.com/Semantic_Segmentation/]
	\label{cocka}
\end{figure}

\chapter{Problem statements}
The assignment of this thesis consists of several expected achievements. Firstly, a promising segmentation method using CNNs needs to be found and implemented. It is expected that the neural network will be as straightforward as possible while still being likely to be capable of giving satisfactory results for the chosen use case (segmentation of a path in outdoor environment for a robot navigation). The images will be provided by the supervisor of the thesis and used to train and validate the network performance. In addition, the author will pick an appropriate software tool for creating Ground Truths (FOOTNOTE manually created image labels that serve as a reference for the network to validate its current accuracy of prediction and to compute the needed adjustments of its parameters to get closer to the desired output) and use it to create the final training and validating datasets. Lastly, the network should be trained with various sets of hyperparameters (FOOTNOTE: hyperparameter definition) in order to get a closer idea of the network's training behaviour and to ensure the best possible results. 

