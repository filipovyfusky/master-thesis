\chapter{Introduction}
Image segmentation is one of the fundamental tasks in computer vision alongside with object recognition and detection. In semantic segmentation, the goal is to assign each pixel of the image a specific category. The difference from image classification is that we do not classify the image as a whole but instead each individual pixel has its own class. 

We can see a real-world example in Figure 1. Each pixel of the image has been assigned to a specific label and represented by a different color. Red for people, blue for cars, green for trees etc.

It is important to say that semantic segmentation is different from so called instance segmentation in which we distinguish labels for instances of the same class. In that case, the people (each instance of the 'person' class) will all have a different color. %[https://theaisummer.com/Semantic_Segmentation/]

It turns out that semantic segmentation has many different applications such as autonomous vehicles, human-computer interaction, robotics, and photo editing/creativity tools. For instance, semantic segmentation is very crucial in self-driving cars and robotics because it is important for the models to understand the context in the environment in which they’re operating. %[https://heartbeat.fritz.ai/a-2019-guide-to-semantic-segmentation-ca8242f5a7fc]

\vspace{5mm}
\begin{figure}[htb]
	\begin{center}
		\includegraphics*[width=13cm, keepaspectratio]{obr/semseg.jpg}
	\end{center}
	\caption{Segmentation of an urban road scene} %[https://theaisummer.com/Semantic_Segmentation/]
	\label{cocka}
\end{figure}

