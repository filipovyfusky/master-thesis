\clearpage
\chapter{Conclusion and future work}

This thesis presented some of the most recent ANN architectures used for image segmentation. In the implementation part, the extensive step-by-step procedure for setting up both the software and hardware environments for running the Caffe implementation of the networks was described and tested on a fresh installation of the operating system. Part of this was showing the benefits of using Debian based distributions of Linux for working with libraries for ANN where the procedure described by shell commands is very clear and can be easily repeated on a different machine. 

The Caffe implementation and auxiliary Python scripts for the presented networks were tuned for the purpose of the thesis. The goal was to perform segmentation on a custom dataset with two object classes. The dataset was created using the best currently available online annotation tool (Labelbox) and is available online. In the training phase, the network architecture was adapted for various transfer learning strategies and showed the power of using pre-trained encoders when the dataset is small. The training hyperparameters were tuned according to the common strategies. As the result, all SegNet variant were successfully trained using AdaDelta optimization and achieve very good values of segmentation accuracy: over 90 \% IOU on the \textit{test} dataset. There is always room for tuning of hyperparameters and achieve even better values of the loss function. 

During the inference phase, the performance of the various architectures was observed and compared. This can give an idea for the computational power needed for further implementations. The probabilistic variants of SegNet can estimate the overall model uncertainty which and hence support the decision making when the network is used in practical applications, such as self-driving robots.
