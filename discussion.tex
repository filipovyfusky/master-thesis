\clearpage
\chapter{Conclusion and Future Work}

This thesis presented some of the most recent ANN architectures used for image segmentation together with their Caffe implementations. An extensive step-by-step procedure for setting up the software and hardware environments was described and tested on a fresh installation of Ubuntu. Part of the reason for this was to show the benefits of using Debian based distributions of Linux for working with ANN libraries: the procedure described by shell commands is very clear and can be easily repeated on a different machine. 

The Caffe implementation and auxiliary Python scripts for the presented networks were tuned for the purpose of the thesis. The goal was to perform a segmentation on a custom dataset with two object classes. The dataset consisting of more than 2600 images was created using the best currently available online annotation tool (Labelbox). In the training phase, the networks were adapted for various transfer learning strategies and showed the power of using pre-trained encoders when the dataset is small. The training hyperparameters were tuned according to common strategies. As a result, all SegNet variants were successfully trained using AdaDelta optimization and achieved very good values of segmentation accuracy: over 90 \% IoU on the \textit{test} dataset. There is always room for further tuning of hyperparameters and expanding the dataset.

The performance of the various architectures was observed and compared during the inference phase. This gives an idea of the computational power needed for further implementations. The probabilistic variants of SegNet can estimate the overall model uncertainty which helps decision making when the network is used in practical applications, such as self-driving robots.
