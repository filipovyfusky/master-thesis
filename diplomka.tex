\documentclass[a4paper,twoside,12pt]{report}% dvoustranný tisk
%\documentclass[12pt]{report}% jednostranný tisk
% všechny soubory jsou v utf-8
	\usepackage{ucs}% pro kódování UTF-8
	\PrerenderUnicode{ěščřžýáíéĚŠČŘŽÝÁÍÉďťňĎŤŇůúÚóÓ} % předkreslení diakritiky, možno přidat/ubrat znaky podle potřeby							                  

\usepackage[czech,english]{babel}
\usepackage[IL2]{fontenc}% csr fonty (pokud jsou nainstalovány česká postscriptová mísma)
%\usepackage[T1]{fontenc}% EC fonty - háčky a čárky jsou k písmenkům připojovány - nehezké

\usepackage{amssymb}
\usepackage{amsmath}
\usepackage[]{csquotes}
\usepackage{enumitem}
\usepackage{subcaption}
\usepackage{fancyhdr}
\usepackage{bm,array}
\usepackage{url}
\usepackage{color}
\usepackage[usenames,dvipsnames,svgnames,table]{xcolor}

\usepackage[nottoc,notlot,notlof, numbib]{tocbibind}

\usepackage[]{diplomka}
\usepackage[]{VSKP} % Sablona dle smernice rektora
%%
%% Styl pro psani Vysokoskolskych kvalifikacnich praci
%% VUT v Brně
%%
%% Pavel Micek, micek@fme.vutbr.cz
%% Jakub Zlamal, zlamal@fme.vutbr.cz
%%
%%
%% Testováno na LuaLaTeX, TeXLive, 2016
%%
\NeedsTeXFormat{LaTeX2e}
\ProvidesPackage{VSKP}[2017/04/26 v3.0 VSKP VUT v Brně]
\RequirePackage{fontspec}


%\def\fakultazkr#1{\edef\fakultazkrtxt{#1}}%fakulta - zkratka fakulty uz neni potreba
\def\fakulta#1{\edef\fakultatxt{#1}\edef\Fakultatxt{\uppercase{#1}}}%fakulta
\def\enfakulta#1{\edef\enfakultatxt{#1}\edef\Enfakultatxt{\uppercase{#1}}}%fakulta
\def\ustav#1{\edef\ustavtxt{#1}\edef\Ustavtxt{\uppercase{#1}}}%ustav
\def\enustav#1{\edef\enustavtxt{#1}\edef\Enustavtxt{\uppercase{#1}}}%ustav
\def\adresafakulta#1{\edef\adresafakultatxt{#1}\edef\Adresafakultatxt{\uppercase{#1}}}%fakulta
\def\logo#1{\def\logotxt{#1}}
\def\logocompletetxt{logo/VUT}

\def\nic{}
\def\nazev#1{
		\edef\NAZEVtxt{\uppercase{#1}}
    \def\nazevtxt{\let\break\nic\let\hfil\nic#1}
    \def\Nazevtxt{\let\break\nic\let\hfil\nic\NAZEVtxt}
    \def\nazevbrtxt{#1}\def\Nazevbrtxt{\NAZEVtxt}}% nazev
\def\ennazev#1{
		\edef\ENNAZEVtxt{\uppercase{#1}}
    \def\ennazevtxt{\let\break\nic\let\hfil\nic#1}
    \def\Ennazevtxt{\let\break\nic\let\hfil\nic\ENNAZEVtxt}
    \def\ennazevbrtxt{#1}\def\Ennazevbrtxt{\ENNAZEVtxt}}% nazev
\def\autor#1#2#3{\edef\autortxt{#1 #2#3}\edef\Autortxt{#1 \uppercase{#2}#3 }}% autor
\def\autorzkr#1{\edef\autorzkrtxt{#1} \edef\Autorzkrtxt{\uppercase{#1}} }% autor zkracene
\def\vedouci#1#2#3{\edef\vedoucitxt{#1 #2#3}\edef\Vedoucitxt{#1 \uppercase{#2}#3 }}% vedouci
\def\ustav#1{\edef\ustavtxt{#1}\edef\Ustavtxt{\uppercase{#1}}}%
\def\datumobhajoby#1{\edef\datumobhajobytxt{#1}}% datum obhajoby
\def\adresa#1{\edef\adresatxt{#1}}% adresa
\def\narozeni#1{\edef\narozenitxt{#1}}% narozeni
\def\muzzena#1{\edef\muzzenatxt{#1}}% muž žena
%%% \def\typprace#1{\edef\typpracetxt{#1}\edef\Typpracetxt{\uppercase{#1}}}% typprace
\long\def\abstrakt#1{\edef\abstrakttxt{#1}}%
\long\def\enabstrakt#1{\edef\enabstrakttxt{#1}}%
\def\klicovaslova#1{\edef\klicovaslovatxt{#1}}%
\def\enklicovaslova#1{\edef\enklicovaslovatxt{#1}}%
\def\citacevedouci#1{\edef\citacevedoucitxt{#1}}% Vedouci dipl. prace do citace
% Fonty VUT pro titulní stranu jsou nakopírované v adresáři ./fonty/
\def\vafle{\fontspec[
	Path			=  fonty/,
	BoldFont 		= Vafle_VUT_Bold.otf ]{Vafle_VUT_Regular.otf}} % Font uložený v adr. fonty
% Velikosti fontů pro titulní stranu
\def\Vutbig{\fontsize{25pt}{30pt}\selectfont} % VUT, 25pt
\def\Vutlar{\fontsize{20pt}{24pt}\selectfont} % Název práce, 20pt
\def\Vutmid{\fontsize{18pt}{22pt}\selectfont} % Fakulta, ustav, 18pt
\def\Vutsmall{\fontsize{13pt}{22pt}\selectfont} % Typ práce, autor, místo, 13pt
\def\Vutsub{\fontsize{11pt}{13pt}\selectfont\textcolor{gray}} % Podtituly, 11pt
\def\Vutsubs{\fontsize{10pt}{12pt}\selectfont\textcolor{gray}} % Podtituly mensi, 10pt
% Mezery mezi texty na titulní straně - standardní
\def\Vutska{\vskip13mm} % Nad VUT
\def\Vutskb{\vskip21mm} % Nad nazvem
\def\Vutskc{\vskip8mm} % Nad fakultou, nad ústavem
\def\Vutskd{\vskip11mm} % Nad typem práce
\def\Vutske{\vskip7mm} % Nad autorem a vedoucím
\def\Vutskf{\vskip13mm} % Nad místem/rokem
% Mezery mezi texty na titulní straně - zúženo, pokud místo a rok přetékají na druhou stranu
%\def\Vutska{\vskip11mm} % Nad VUT
%\def\Vutskb{\vskip21mm} % Nad nazvem
%\def\Vutskc{\vskip8mm} % Nad fakultou, nad ústavem
%\def\Vutskd{\vskip11mm} % Nad typem práce
%\def\Vutske{\vskip7mm} % Nad autorem a vedoucím
%\def\Vutskf{\vskip13mm} % Nad místem/rokem

% Prevod typu studia -> typ prace,
% v ciselniku se lisi en nazev u M a N typu studia
\def\typstudia#1{%
    \def\tstudia{#1}
    \def\vzortypu{N}
    \ifx\tstudia\vzortypu
        \edef\typpracetxt{master's thesis}
        \edef\entyppracetxt{diplomová práce}
    \else
        \edef\typpracetxt{neznámý typ studia: \tstudia}
        \edef\entyppracetxt{unknown studium type: \tstudia}
    \fi
    \def\vzortypu{B}
    \ifx\tstudia\vzortypu
        \edef\typpracetxt{bakalářská práce}
        \edef\entyppracetxt{bachelor's thesis}
    \fi
    \def\vzortypu{D}
    \ifx\tstudia\vzortypu
        \edef\typpracetxt{dizertační práce}
        \edef\entyppracetxt{doctoral thesis}
    \fi
    \def\vzortypu{M}
    \ifx\tstudia\vzortypu
        \edef\typpracetxt{diplomová práce}
        \edef\entyppracetxt{diploma thesis}
    \fi
    \edef\Typpracetxt{\uppercase{\typpracetxt}}
    \edef\Entyppracetxt{\uppercase{\entyppracetxt}}
}


%
%  vytvoreni titulni strany
%
\def\titul{%
\setcounter{page}{1}
\thispagestyle{empty}

\long\def\obrbeztextu##1{%
   \setbox1=\hbox{\includegraphics*[width=41mm,height=41mm,keepaspectratio]{##1}}
   \dimen0=\pagewidth
   \advance\dimen0 by -\oddsidemargin
   \advance\dimen0 by -\wd1
   \advance\dimen0 by -0.5em
   \setbox2=\vbox to \ht1{\hsize=\dimen0%
   \vss
   }
   \wd2=\dimen0
   \noindent\hbox to \hsize{\box1\hskip0.5em\box2\hss}
}

{\flushleft

\obrbeztextu{\logocompletetxt}
% Mezery níže možno upravit - tyto jsou nastaveny tak, aby se i při třířádkovém názvu 
% práce a dvouřádkovém názvu ústavu vysázel poslední řádek na správnou stranu
\Vutska
\vafle
\noindent{\Vutbig BRNO UNIVERSITY OF TECHNOLOGY\hfil}\par
\smallskip
\noindent{\Vutsub{VYSOKÉ UČENÍ TECHNICKÉ V BRNĚ}}
\bgroup
\Vutskc
   \noindent{\Vutmid\Fakultatxt}\par
   \smallskip
   \noindent{\Vutsub\Enfakultatxt}

   \Vutskc

   \noindent{\Vutmid\Ustavtxt}\par
   \smallskip
   \noindent{\Vutsub\Enustavtxt}

\Vutskb
{\advance\baselineskip by 6pt
\noindent{\Vutlar\Nazevbrtxt }

}

\noindent{\Vutsub\Ennazevbrtxt }

\Vutskd
\noindent{\Vutsmall\Typpracetxt}

\noindent{\Vutsubs\Entyppracetxt}

\Vutske
% troska rozhodovani jak vytisknout autora a suprevizora
\setbox1=\hbox{\Vutsmall\Autortxt}
\setbox2=\hbox{\Vutsmall\Vedoucitxt}
\dimen0=\textwidth
\advance\dimen0 by -7cm%  5+2cm
\ifdim\wd1>\wd2
   \dimen1=\wd1
\else
   \dimen1=\wd2
\fi
\ifdim\dimen0<\dimen1
   % nevejde se mi tam jmeno autora nebo vedouciho protoze je moc dlouhe
   \noindent\Vutsmall{AUTHOR\hfill\hbox to \dimen1{\Autortxt\hss}}

\noindent\Vutsubs{AUTOR PRÁCE}

\Vutske
\noindent\Vutsmall{SUPERVISOR\hfill\hbox to \dimen1{\Vedoucitxt\hss}}

\noindent{\Vutsubs{VEDOUCÍ PRÁCE}}
\else
   \noindent{\hbox to 5cm{AUTHOR\hss}\hskip2cm\Autortxt}

   \noindent{\Vutsubs{AUTOR PRÁCE}}

   \Vutske
   \noindent{\hbox to 5cm{SUPERVISOR\hss}\hskip2cm\Vedoucitxt}

   \noindent{\Vutsubs{VEDOUCÍ PRÁCE}}
\fi

\Vutskf
\noindent{\Vutsmall BRNO \the\year}
}
\egroup % End of flushleft
\eject
}

%
% abstrakty a klicova slova
%
\def\abstrakty{%
   \newpage
   \thispagestyle{empty}

\noindent{\bf Abstrakt}

\abstrakttxt
\bigskip

\noindent{\bf Summary}

\enabstrakttxt

\bigskip
\bigskip
%\goodbreak
\vbox{
\noindent{\bf Klíčová slova}

\klicovaslovatxt

\bigskip
\noindent{\bf Keywords}

\enklicovaslovatxt
}
\nobreak
\vfill 
% Nasleduje ukazkova citace diplomove prace
\def\prvnivelke##1##2{\uppercase{##1}##2}
\noindent \Autorzkrtxt {\it \nazevtxt}. Brno: Vysoké učení technické v Brně, \fakultatxt, \the\year. %
\ifx\pocetstran\undefined
   ??
\else
   \pocetstran{}
\fi s. \citacevedoucitxt
\vskip3cm
\eject
}

\long\def\prohlaseni#1{
\bgroup
\pagestyle{empty}
\cleardoublepage
\egroup
\thispagestyle{empty}
\hbox{}
\vfill
#1
\bigskip
\bigskip

\hfill\autortxt\hskip3cm
\vskip2cm
\eject
}

\long\def\podekovani#1{
\bgroup
\pagestyle{empty}
\cleardoublepage
\egroup
\thispagestyle{empty}
\hbox{}
\vfill
#1
\bigskip
\bigskip

\hfill\autortxt\hskip3cm
\vskip2cm
\eject
}

%https://tex.stackexchange.com/questions/270261/give-number-one-to-a-left-page-without-making-it-a-right-page
\def\obsah{
%	\makeatletter
%	\renewcommand{\thepage}{\@arabic{\numexpr\value{page}-1}}
%	\makeatother	
	\setcounter{page}{1}\tableofcontents\vfill\eject
}

\makeatletter
\def\spocitejstranky{
\protected@write\@auxout{}{\string\gdef\string\pocetstran{\thepage}}%
}
\makeatother

\AtEndDocument{\spocitejstranky}
 % Uvodni desky atd dle smernice rektora
\splithyphens% při rozdělování slov se spojovníkem opakuj spojovník
\usepackage[pdftitle={\typpracetxt},
            pdfauthor={\autortxt},
            bookmarks=true,
            pdfencoding=unicode,
            linkcolor=blue,
            colorlinks=true,
            breaklinks=true]{hyperref}
            
\hypersetup{
  citecolor=Black,
  linkcolor=Black,
  urlcolor=Blue}
  
%\usepackage[pdftex]{graphicx}
% Pro vytvoření titulního listu je potreba další balíček
\usepackage{fontspec}  % Pro vkládání OTF fontů (vyžaduje titulní list) - nefunguje v pdfLaTeXu
% Pro vložení titulního listu staženého ze Studisu stačí jen vkládáni PDF
\usepackage{pdfpages} % Pro vkladání PDF souborů (s titulním listem apod.)
\DeclareGraphicsExtensions{.png,.pdf}

\begin{document}

\titul% vytiskne titul práce
\abstrakty% vytiskne stránku s abstrakty


\prohlaseni{Prohlašuji, že jsem svou práci vypracoval samostatně a použil jsem pouze podklady (literaturu, software atd.) citované v práci a uvedené v přiloženém seznamu a postup při zpracování práce je v souladu se zákonem č. 121/2000\,Sb.,o právu autorském, o právech souvisejících s právem autorským a o změně některých zákonů (autorský zákon) v platném znění. \newline

\noindent V Brně 1. května 2017
}% prohlášení,

\podekovani{Děkuji Ing. Tomáši Spáčilovi, Matěji Rajchlovi a celému týmu z Mechlabu za podnětné připomínky a rady, které mi během práce poskytli. Dále děkuji své rodině za podporu jak během studií, tak během psaní této práce.

}% poděkování, nepovinné

% =======================================vlastní práce==========================================
\obsah% vytiskne obsah

%
%  vlastni text
%
\chapter{Introduction}
Image segmentation is one of the fundamental tasks in computer vision alongside with object recognition and detection. In semantic segmentation, the goal is to assign each pixel of the image a specific category. The difference from image classification is that we do not classify the image as a whole but instead each individual pixel has its own class. 

We can see a real-world example in Figure 1. Each pixel of the image has been assigned to a specific label and represented by a different color. Red for people, blue for cars, green for trees etc.

It is important to say that semantic segmentation is different from so called instance segmentation in which we distinguish labels for instances of the same class. In that case, the people (each instance of the 'person' class) will all have a different color. %[https://theaisummer.com/Semantic_Segmentation/]

It turns out that semantic segmentation has many different applications such as autonomous vehicles, human-computer interaction, robotics, and photo editing/creativity tools. For instance, semantic segmentation is very crucial in self-driving cars and robotics because it is important for the models to understand the context in the environment in which they’re operating. %[https://heartbeat.fritz.ai/a-2019-guide-to-semantic-segmentation-ca8242f5a7fc]

\vspace{5mm}
\begin{figure}[htb]
	\begin{center}
		\includegraphics*[width=13cm, keepaspectratio]{obr/semseg.jpg}
	\end{center}
	\caption{Segmentation of an urban road scene} %[https://theaisummer.com/Semantic_Segmentation/]
	\label{cocka}
\end{figure}

% nutné
%
% sem vlastni opsany text, možno vložit více souborů (nejlépe pro každou kapitolu zvláštní soubor)
\chapter{Research and theory}
\label{research}


%
%\chapter{Závěr}
Závěr je opravdu nutný  rozhodně delší než jeden řádek. Závěr je opravdu nutný  rozhodně delší než jeden řádek. Závěr je opravdu nutný  rozhodně delší než jeden řádek. Závěr je opravdu nutný  rozhodně delší než jeden řádek. Závěr je opravdu nutný  rozhodně delší než jeden řádek. Závěr je opravdu nutný  rozhodně delší než jeden řádek. Závěr je opravdu nutný  rozhodně delší než jeden řádek. 
% nutné
\begin{thebibliography}{99}

\bibitem{mwiti}{MWITI, Derrick. 2019. A 2019 Guide to Semantic Segmentation. In: \textit{Heartbeat} [online]. Fritz AI. Available at: https://heartbeat.fritz.ai/a-2019-guide-to-semantic-segmentation-ca8242f5a7fc}

\bibitem{sergios}{KARAGIANNAKOS, Sergios. 2019. Semantic Segmentation in the era of Neural Networks. In: \textit{AI SUMMER} [online]. Available at: https://theaisummer.com/Semantic\_Segmentation/}

\bibitem{mehlig}{MEHLIG, Bernhard. 2019. Artificial Neural Networks. ArXiv.org [online]. Available at: https://arxiv.org/abs/1901.05639}

\bibitem{santiago}{PUENTE, Santiago. 2018. \textit{Single and Multi-Label Environmental Sound Classification Using Convolutional Neural Networks} [online]. Gothenburg. Available at: https://odr.chalmers.se/handle/20.500.12380/255604. Master's thesis. Chalmers University of Technology.}

\bibitem{goodfellow}{GOODFELLOW, Ian, Yoshua BENGIO a Aaron COURVILLE. \textit{Deep learning}. Cambridge, Massachusetts: The MIT Press, 2016. ISBN 978-026-2035-613.}

\bibitem{groman}{GROMAN, Martin. Tvorba umělé neuronové sítě pro výpočet termodynamických veličin [online]. Brno, 2019 [cit. 2020-06-07]. Available at: \url{http://hdl.handle.net/11012/175381}. Master's thesis. Vysoké učení technické v Brně. Fakulta strojního inženýrství. Ústav matematiky. Supervisor Tomáš Mauder.}

\bibitem{stanford-github}{CS231n: Convolutional Neural Networks for Visual Recognition: Lecture Notes. \textit{CS231n: Convolutional Neural Networks for Visual Recognition} [online]. Stanford: Stanford University. Available at: https://cs231n.github.io/}

\bibitem{stanford-L4}{Lecture 4|Introduction to Neural Networks \textit{YouTube} [online]. 11. August 2018. Available at: \url{https://www.youtube.com/watch?v=d14TUNcbn1k&list=PL3FW7Lu3i5JvHM8ljYj-zLfQRF3EO8sYv&index=4} }

\bibitem{stanford-L6}{Lecture 6|Training Neural Networks I \textit{YouTube} [online]. 11. August 2018. Available at: \url{https://www.youtube.com/watch?v=wEoyxE0GP2M&list=PL3FW7Lu3i5JvHM8ljYj-zLfQRF3EO8sYv&index=6} }

\bibitem{stanford-L7}{Lecture 7|Training Neural Networks II \textit{YouTube} [online]. 11. August 2018. Available at: \url{https://www.youtube.com/watch?v=_JB0AO7QxSA&list=PL3FW7Lu3i5JvHM8ljYj-zLfQRF3EO8sYv&index=7} }

\bibitem{coors}{COORS, Benjamin. 2016. \textit{Navigation of Mobile Robots in Human Environments with Deep Reinforcement Learning} [online]. Stockholm. Available at: \url{http://www.diva-portal.org/smash/record.jsf?pid=diva2\%3A967644&dswid=9005. Degree project. KTH Royal Institute of Technology.}}
		
\bibitem{eniola}{ALESE, Eniola. 2018. The curious case of the vanishing and exploding gradient. In: \textit{Medium} [online]. Available at: https://medium.com/learn-love-ai/the-curious-case-of-the-vanishing-exploding-gradient-bf58ec6822eb}

\bibitem{bushaev}{BUSHAEV, Vitaly. 2018. Adam — latest trends in deep learning optimization. In: Towards Data Science [online]. Towards Data Science. Available at: \url{https://towardsdatascience.com/adam-latest-trends-in-deep-learning-optimization-6be9a291375c}}

\bibitem{issue}{Batch Normalization Issue in SegNet. 2017. In: \textit{GitHub} [online]. GitHub. Available at: https://github.com/alexgkendall/caffe-segnet/issues/109}

\bibitem{arvi}{JONNARTH, Arvi. 2018. \textit{Camera-Based Friction Estimation with Deep Convolutional Neural Networks} [online]. Uppsala. Available at: https://pdfs.semanticscholar.org/4c35/becacb2aab803468eb38f19d8418d79c7c08.pdf. Master's thesis. Uppsala Universitet.}

\bibitem{krizhevsky}{KRIZHEVSKY, Alex, Ilya SUTSKEVER and Geoffrey HINTON. 2017. ImageNet classification with deep convolutional neural networks. \textit{Communications of the ACM} [online]. ACM. Available at: \url{http://web.b.ebscohost.com.ezproxy.lib.vutbr.cz/ehost/detail/detail?vid=0&sid=dcac7028-11f2-41e4-ba7d-d517a5a51a6f%40sessionmgr101&bdata=Jmxhbmc9Y3Mmc2l0ZT1laG9zdC1saXZl#AN=123446102&db=bth}}
		
\bibitem{lecun}{LECUN, Y, L BOTTOU, Y BENGIO and P HAFFNER. 1998. Gradient-based learning applied to document recognition. \textit{Proceedings of the IEEE} [online]. IEEE, 86(11), 2278-2324. Available at: \url{https://ieeexplore-ieee-org.ezproxy.lib.vutbr.cz/document/726791}}

\bibitem{szegedy}{SZEGEDY, Christian, Wei LIU, Yangqing JIA, Pierre SERMANET, Scott REED, Dragomir ANGUELOV, Vincent VANHOUCKE and Andrew RABINOVICH. 2014. Going Deeper with Convolutions. \textit{ArXiv.org} [online]. Ithaca: Cornell University Library, arXiv.org. Available at: \url{http://search.proquest.com/docview/2084489417/}}

\bibitem{vgg}{SIMONYAN, Karen and Andrew ZISSERMAN. 2014. Very Deep Convolutional Networks for Large-Scale Visual Recognition. \textit{Visual Geometry Group} [online]. Oxford: University of Oxford. Available at: \url{http://www.robots.ox.ac.uk/~vgg/research/very_deep/}}

\bibitem{resnet}{HE, Kaiming, Xiangyu ZHANG, Shaoqing REN and Jian SUN. 2015. Deep Residual Learning for Image Recognition. \textit{ArXiv.org} [online]. Ithaca: Cornell University Library, arXiv.org. Available at: \url{http://search.proquest.com/docview/2083823373}}

\bibitem{coufal}{COUFAL, J. Detekce cesty pro mobilní robot analýzou obrazu. Brno: Vysoké učení
	technické v Brně, Fakulta strojního inženýrství, 2010. 49 s. Master's thesis. Supervisor: Ing.
	Jiří Krejsa, Ph.D}

\bibitem{bayesian}{KENDALL, Alex, Vijay BADRINARAYANAN and Roberto CIPOLLA. 2016. Bayesian SegNet: Model Uncertainty in Deep Convolutional Encoder-Decoder Architectures for Scene Understanding. \textit{ArXiv.org} [online]. Ithaca: Cornell University Library, arXiv.org. Available at: \url{https://arxiv.org/abs/1511.02680}}

\bibitem{segnet}{BADRINARAYANAN, Vijay, Alex KENDALL and Roberto CIPOLLA. 2016. SegNet: A Deep Convolutional Encoder-Decoder Architecture for Image Segmentation.\textit{ ArXiv.org} [online]. Ithaca: Cornell University Library, arXiv.org. Available at: \url{https://arxiv.org/abs/1511.00561}}

\bibitem{segnet_tut}{KENDALL, Alex, Vijay BADRINARAYANAN and Roberto CIPOLLA. 2015. SegNet. \textit{Machine Intelligence Laboratory} [online]. Cambridge: University of Cambridge. Available at: \url{https://mi.eng.cam.ac.uk/projects/segnet/}}

\bibitem{zeltner}{ZELTNER, Felix. 2016. Autonomous Terrain Classification Through Unsupervised Learning [online]. Luleå. Available at: \url{http://ltu.diva-portal.org/smash/record.jsf?pid=diva2%3A1051763&dswid=-6301. Degree project. Luleå University of Technology.}}
	
\bibitem{segnet_get_started}{KENDALL, Alex. 2015. Getting Started with SegNet. \textit{Machine Intelligence Laboratory} [online]. Cambridge: University of Cambridge. Available at: \url{http://mi.eng.cam.ac.uk/projects/segnet/tutorial.html}}

\bibitem{iou}{ROSEBROCK, Adrian. 2016. Intersection over Union (IoU) for object detection. In: \textit{Pyimagesearch} [online]. pyimagesearch. Available at: \url{https://www.pyimagesearch.com/2016/11/07/intersection-over-union-iou-for-object-detection/}}

\bibitem{filip_github}{SegNet-Tutorial. 2020. \textit{GitHub} [online]. GitHub. Available at: \url{https://github.com/filipovyfusky/SegNet-Tutorial}}

\bibitem{nvidia}{\textit{NVIDIA} [online]. 2020. USA: NVIDIA. Available at: https://www.nvidia.com/}

\bibitem{nvidia_dev}{\textit{NVIDIA Developer} [online]. 2020. USA: NVIDIA. Available at: https://developer.nvidia.com/}

\bibitem{caffe}{\textit{Caffe} [online]. Berkeley: Berkeley AI Reseach. Available at: https://caffe.berkeleyvision.org/}

\bibitem{filip_github_caffe}{caffe-segnet-cudnn5. 2020. \textit{GitHub} [online]. GitHub. Available at: \url{https://github.com/filipovyfusky/caffe-segnet-cudnn5}}

\bibitem{aizawan_github}{SegNet implementation in Tensorflow. 2020. \textit{GitHub} [online]. GitHub. Available at: \url{https://github.com/aizawan/segnet}}

\bibitem{labelbox}{\textit{Labelbox} [online]. 2020. Available at: https://labelbox.com}

\bibitem{theano}{Convolution arithmetic tutorial. 2018. \textit{Deep Learning} [online]. LISA lab. Available at: \url{http://deeplearning.net/software/theano_versions/dev/tutorial/conv_arithmetic.html}}

\bibitem{gigabyte}{\textit{GIGABYTE} [online]. 2020. GIGABYTE. Available at: https://www.gigabyte.com/}

\end{thebibliography}
% nutné
\chapter*{List of Abbreviations}
\addcontentsline{toc}{chapter}{List of Abbreviations} 
\symbolsize=3.5cm% sirka sloupecku pro symboly, je mozno zmensit pokud jsou kratke
\vspace{7mm}
\begin{symboly}

\item[CNN] Convolutional Neural Network
\item[OS] Operating System
\item[ANN] Artificial Neural Network
\item[ReLU] Rectified Linear Unit
\item[MSE] Mean Squared Error
\item[SGD] Stochastic Gradient Descent
\item[RGB] Red-Green-Blue
\item[FCN] Fully Connected Network
\item[MCDO] Monte Carlo Dropout
\item[IoU] Intersection over Union
\item[CPU] Central Processing Unit
\item[GPU] Graphics Processing Unit
\item[AI] Artificial Intelligence
\item[BAIR] Berkeley AI Research
\item[CPW] Caffe Python Wrapper
\item[SSD] Solid-State Drive
\item[RAM] Random-Access Memory
\item[LTS] Long-Term Support

\end{symboly}
% nutné
\chapter*{List of Attachments}
\addcontentsline{toc}{chapter}{List of Attachments} 
\vspace{7mm}
\begin{enumerate}
\item Architecture of SegNet (all variants): \textit{scheme.pdf}

\end{enumerate}

% není povinné
\end{document}
