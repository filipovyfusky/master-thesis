\documentclass[a4paper,twoside,12pt]{report}% dvoustranný tisk
%\documentclass[12pt]{report}% jednostranný tisk
% všechny soubory jsou v utf-8
	\usepackage{ucs}% pro kódování UTF-8
	\PrerenderUnicode{ěščřžýáíéĚŠČŘŽÝÁÍÉďťňĎŤŇůúÚóÓ} % předkreslení diakritiky, možno přidat/ubrat znaky podle potřeby							                  

\usepackage[czech,english]{babel}
\usepackage[IL2]{fontenc}% csr fonty (pokud jsou nainstalovány česká postscriptová mísma)
%\usepackage[T1]{fontenc}% EC fonty - háčky a čárky jsou k písmenkům připojovány - nehezké


\usepackage{listings}
\usepackage{algorithmicx}
\usepackage{algorithm}
\usepackage{amssymb}
\usepackage{amsmath}
\usepackage[]{csquotes}
\usepackage{enumitem}
\usepackage{subcaption}
\usepackage{fancyhdr}
\usepackage{bm,array}
\usepackage{url}
\usepackage{color}
\usepackage[usenames,dvipsnames,svgnames,table]{xcolor}

\usepackage[nottoc,notlot,notlof, numbib]{tocbibind}

\usepackage[]{diplomka}
\usepackage[]{VSKP} % Sablona dle smernice rektora

\definecolor{dkgreen}{rgb}{0,0.6,0}
\definecolor{gray}{rgb}{0.5,0.5,0.5}
\definecolor{mauve}{rgb}{0.58,0,0.82}

\lstset{frame=tb,
	language=C++,
	aboveskip=3mm,
	belowskip=3mm,
	showstringspaces=false,
	columns=flexible,
	basicstyle={\small\ttfamily},
	numbers=none,
	numberstyle=\tiny\color{gray},
	keywordstyle=\color{blue},
	commentstyle=\color{dkgreen},
	stringstyle=\color{mauve},
	breakatwhitespace=true,
	tabsize=3,
	breaklines=true
}

%%
%% Styl pro psani Vysokoskolskych kvalifikacnich praci
%% VUT v Brně
%%
%% Pavel Micek, micek@fme.vutbr.cz
%% Jakub Zlamal, zlamal@fme.vutbr.cz
%%
%%
%% Testováno na LuaLaTeX, TeXLive, 2016
%%
\NeedsTeXFormat{LaTeX2e}
\ProvidesPackage{VSKP}[2017/04/26 v3.0 VSKP VUT v Brně]
\RequirePackage{fontspec}


%\def\fakultazkr#1{\edef\fakultazkrtxt{#1}}%fakulta - zkratka fakulty uz neni potreba
\def\fakulta#1{\edef\fakultatxt{#1}\edef\Fakultatxt{\uppercase{#1}}}%fakulta
\def\enfakulta#1{\edef\enfakultatxt{#1}\edef\Enfakultatxt{\uppercase{#1}}}%fakulta
\def\ustav#1{\edef\ustavtxt{#1}\edef\Ustavtxt{\uppercase{#1}}}%ustav
\def\enustav#1{\edef\enustavtxt{#1}\edef\Enustavtxt{\uppercase{#1}}}%ustav
\def\adresafakulta#1{\edef\adresafakultatxt{#1}\edef\Adresafakultatxt{\uppercase{#1}}}%fakulta
\def\logo#1{\def\logotxt{#1}}
\def\logocompletetxt{logo/VUT}

\def\nic{}
\def\nazev#1{
		\edef\NAZEVtxt{\uppercase{#1}}
    \def\nazevtxt{\let\break\nic\let\hfil\nic#1}
    \def\Nazevtxt{\let\break\nic\let\hfil\nic\NAZEVtxt}
    \def\nazevbrtxt{#1}\def\Nazevbrtxt{\NAZEVtxt}}% nazev
\def\ennazev#1{
		\edef\ENNAZEVtxt{\uppercase{#1}}
    \def\ennazevtxt{\let\break\nic\let\hfil\nic#1}
    \def\Ennazevtxt{\let\break\nic\let\hfil\nic\ENNAZEVtxt}
    \def\ennazevbrtxt{#1}\def\Ennazevbrtxt{\ENNAZEVtxt}}% nazev
\def\autor#1#2#3{\edef\autortxt{#1 #2#3}\edef\Autortxt{#1 \uppercase{#2}#3 }}% autor
\def\autorzkr#1{\edef\autorzkrtxt{#1} \edef\Autorzkrtxt{\uppercase{#1}} }% autor zkracene
\def\vedouci#1#2#3{\edef\vedoucitxt{#1 #2#3}\edef\Vedoucitxt{#1 \uppercase{#2}#3 }}% vedouci
\def\ustav#1{\edef\ustavtxt{#1}\edef\Ustavtxt{\uppercase{#1}}}%
\def\datumobhajoby#1{\edef\datumobhajobytxt{#1}}% datum obhajoby
\def\adresa#1{\edef\adresatxt{#1}}% adresa
\def\narozeni#1{\edef\narozenitxt{#1}}% narozeni
\def\muzzena#1{\edef\muzzenatxt{#1}}% muž žena
%%% \def\typprace#1{\edef\typpracetxt{#1}\edef\Typpracetxt{\uppercase{#1}}}% typprace
\long\def\abstrakt#1{\edef\abstrakttxt{#1}}%
\long\def\enabstrakt#1{\edef\enabstrakttxt{#1}}%
\def\klicovaslova#1{\edef\klicovaslovatxt{#1}}%
\def\enklicovaslova#1{\edef\enklicovaslovatxt{#1}}%
\def\citacevedouci#1{\edef\citacevedoucitxt{#1}}% Vedouci dipl. prace do citace
% Fonty VUT pro titulní stranu jsou nakopírované v adresáři ./fonty/
\def\vafle{\fontspec[
	Path			=  fonty/,
	BoldFont 		= Vafle_VUT_Bold.otf ]{Vafle_VUT_Regular.otf}} % Font uložený v adr. fonty
% Velikosti fontů pro titulní stranu
\def\Vutbig{\fontsize{25pt}{30pt}\selectfont} % VUT, 25pt
\def\Vutlar{\fontsize{20pt}{24pt}\selectfont} % Název práce, 20pt
\def\Vutmid{\fontsize{18pt}{22pt}\selectfont} % Fakulta, ustav, 18pt
\def\Vutsmall{\fontsize{13pt}{22pt}\selectfont} % Typ práce, autor, místo, 13pt
\def\Vutsub{\fontsize{11pt}{13pt}\selectfont\textcolor{gray}} % Podtituly, 11pt
\def\Vutsubs{\fontsize{10pt}{12pt}\selectfont\textcolor{gray}} % Podtituly mensi, 10pt
% Mezery mezi texty na titulní straně - standardní
\def\Vutska{\vskip13mm} % Nad VUT
\def\Vutskb{\vskip21mm} % Nad nazvem
\def\Vutskc{\vskip8mm} % Nad fakultou, nad ústavem
\def\Vutskd{\vskip11mm} % Nad typem práce
\def\Vutske{\vskip7mm} % Nad autorem a vedoucím
\def\Vutskf{\vskip13mm} % Nad místem/rokem
% Mezery mezi texty na titulní straně - zúženo, pokud místo a rok přetékají na druhou stranu
%\def\Vutska{\vskip11mm} % Nad VUT
%\def\Vutskb{\vskip21mm} % Nad nazvem
%\def\Vutskc{\vskip8mm} % Nad fakultou, nad ústavem
%\def\Vutskd{\vskip11mm} % Nad typem práce
%\def\Vutske{\vskip7mm} % Nad autorem a vedoucím
%\def\Vutskf{\vskip13mm} % Nad místem/rokem

% Prevod typu studia -> typ prace,
% v ciselniku se lisi en nazev u M a N typu studia
\def\typstudia#1{%
    \def\tstudia{#1}
    \def\vzortypu{N}
    \ifx\tstudia\vzortypu
        \edef\typpracetxt{master's thesis}
        \edef\entyppracetxt{diplomová práce}
    \else
        \edef\typpracetxt{neznámý typ studia: \tstudia}
        \edef\entyppracetxt{unknown studium type: \tstudia}
    \fi
    \def\vzortypu{B}
    \ifx\tstudia\vzortypu
        \edef\typpracetxt{bakalářská práce}
        \edef\entyppracetxt{bachelor's thesis}
    \fi
    \def\vzortypu{D}
    \ifx\tstudia\vzortypu
        \edef\typpracetxt{dizertační práce}
        \edef\entyppracetxt{doctoral thesis}
    \fi
    \def\vzortypu{M}
    \ifx\tstudia\vzortypu
        \edef\typpracetxt{diplomová práce}
        \edef\entyppracetxt{diploma thesis}
    \fi
    \edef\Typpracetxt{\uppercase{\typpracetxt}}
    \edef\Entyppracetxt{\uppercase{\entyppracetxt}}
}


%
%  vytvoreni titulni strany
%
\def\titul{%
\setcounter{page}{1}
\thispagestyle{empty}

\long\def\obrbeztextu##1{%
   \setbox1=\hbox{\includegraphics*[width=41mm,height=41mm,keepaspectratio]{##1}}
   \dimen0=\pagewidth
   \advance\dimen0 by -\oddsidemargin
   \advance\dimen0 by -\wd1
   \advance\dimen0 by -0.5em
   \setbox2=\vbox to \ht1{\hsize=\dimen0%
   \vss
   }
   \wd2=\dimen0
   \noindent\hbox to \hsize{\box1\hskip0.5em\box2\hss}
}

{\flushleft

\obrbeztextu{\logocompletetxt}
% Mezery níže možno upravit - tyto jsou nastaveny tak, aby se i při třířádkovém názvu 
% práce a dvouřádkovém názvu ústavu vysázel poslední řádek na správnou stranu
\Vutska
\vafle
\noindent{\Vutbig BRNO UNIVERSITY OF TECHNOLOGY\hfil}\par
\smallskip
\noindent{\Vutsub{VYSOKÉ UČENÍ TECHNICKÉ V BRNĚ}}
\bgroup
\Vutskc
   \noindent{\Vutmid\Fakultatxt}\par
   \smallskip
   \noindent{\Vutsub\Enfakultatxt}

   \Vutskc

   \noindent{\Vutmid\Ustavtxt}\par
   \smallskip
   \noindent{\Vutsub\Enustavtxt}

\Vutskb
{\advance\baselineskip by 6pt
\noindent{\Vutlar\Nazevbrtxt }

}

\noindent{\Vutsub\Ennazevbrtxt }

\Vutskd
\noindent{\Vutsmall\Typpracetxt}

\noindent{\Vutsubs\Entyppracetxt}

\Vutske
% troska rozhodovani jak vytisknout autora a suprevizora
\setbox1=\hbox{\Vutsmall\Autortxt}
\setbox2=\hbox{\Vutsmall\Vedoucitxt}
\dimen0=\textwidth
\advance\dimen0 by -7cm%  5+2cm
\ifdim\wd1>\wd2
   \dimen1=\wd1
\else
   \dimen1=\wd2
\fi
\ifdim\dimen0<\dimen1
   % nevejde se mi tam jmeno autora nebo vedouciho protoze je moc dlouhe
   \noindent\Vutsmall{AUTHOR\hfill\hbox to \dimen1{\Autortxt\hss}}

\noindent\Vutsubs{AUTOR PRÁCE}

\Vutske
\noindent\Vutsmall{SUPERVISOR\hfill\hbox to \dimen1{\Vedoucitxt\hss}}

\noindent{\Vutsubs{VEDOUCÍ PRÁCE}}
\else
   \noindent{\hbox to 5cm{AUTHOR\hss}\hskip2cm\Autortxt}

   \noindent{\Vutsubs{AUTOR PRÁCE}}

   \Vutske
   \noindent{\hbox to 5cm{SUPERVISOR\hss}\hskip2cm\Vedoucitxt}

   \noindent{\Vutsubs{VEDOUCÍ PRÁCE}}
\fi

\Vutskf
\noindent{\Vutsmall BRNO \the\year}
}
\egroup % End of flushleft
\eject
}

%
% abstrakty a klicova slova
%
\def\abstrakty{%
   \newpage
   \thispagestyle{empty}

\noindent{\bf Abstrakt}

\abstrakttxt
\bigskip

\noindent{\bf Summary}

\enabstrakttxt

\bigskip
\bigskip
%\goodbreak
\vbox{
\noindent{\bf Klíčová slova}

\klicovaslovatxt

\bigskip
\noindent{\bf Keywords}

\enklicovaslovatxt
}
\nobreak
\vfill 
% Nasleduje ukazkova citace diplomove prace
\def\prvnivelke##1##2{\uppercase{##1}##2}
\noindent \Autorzkrtxt {\it \nazevtxt}. Brno: Vysoké učení technické v Brně, \fakultatxt, \the\year. %
\ifx\pocetstran\undefined
   ??
\else
   \pocetstran{}
\fi s. \citacevedoucitxt
\vskip3cm
\eject
}

\long\def\prohlaseni#1{
\bgroup
\pagestyle{empty}
\cleardoublepage
\egroup
\thispagestyle{empty}
\hbox{}
\vfill
#1
\bigskip
\bigskip

\hfill\autortxt\hskip3cm
\vskip2cm
\eject
}

\long\def\podekovani#1{
\bgroup
\pagestyle{empty}
\cleardoublepage
\egroup
\thispagestyle{empty}
\hbox{}
\vfill
#1
\bigskip
\bigskip

\hfill\autortxt\hskip3cm
\vskip2cm
\eject
}

%https://tex.stackexchange.com/questions/270261/give-number-one-to-a-left-page-without-making-it-a-right-page
\def\obsah{
%	\makeatletter
%	\renewcommand{\thepage}{\@arabic{\numexpr\value{page}-1}}
%	\makeatother	
	\setcounter{page}{1}\tableofcontents\vfill\eject
}

\makeatletter
\def\spocitejstranky{
\protected@write\@auxout{}{\string\gdef\string\pocetstran{\thepage}}%
}
\makeatother

\AtEndDocument{\spocitejstranky}
 % Uvodni desky atd dle smernice rektora
\splithyphens% při rozdělování slov se spojovníkem opakuj spojovník
\usepackage[pdftitle={\typpracetxt},
            pdfauthor={\autortxt},
            bookmarks=true,
            pdfencoding=unicode,
            linkcolor=blue,
            colorlinks=true,
            breaklinks=true]{hyperref}
            
\hypersetup{
  citecolor=Black,
  linkcolor=Black,
  urlcolor=Blue}
  
%\usepackage[pdftex]{graphicx}
% Pro vytvoření titulního listu je potreba další balíček
\usepackage{fontspec}  % Pro vkládání OTF fontů (vyžaduje titulní list) - nefunguje v pdfLaTeXu
% Pro vložení titulního listu staženého ze Studisu stačí jen vkládáni PDF
\usepackage{pdfpages} % Pro vkladání PDF souborů (s titulním listem apod.)
\DeclareGraphicsExtensions{.png,.pdf}

\begin{document}

\titul% vytiskne titul práce
\abstrakty% vytiskne stránku s abstrakty


\prohlaseni{Prohlasuji}% prohlášení,

\podekovani{Dekuji, FS}% poděkování, nepovinné

% =======================================vlastní práce==========================================
\obsah% vytiskne obsah

\chapter{Introduction}
Image segmentation is one of the fundamental tasks in computer vision alongside with object recognition and detection. In semantic segmentation, the goal is to assign each pixel of the image a specific category. The difference from image classification is that we do not classify the image as a whole but instead each individual pixel has its own class. 

We can see a real-world example in Figure 1. Each pixel of the image has been assigned to a specific label and represented by a different color. Red for people, blue for cars, green for trees etc.

It is important to say that semantic segmentation is different from so called instance segmentation in which we distinguish labels for instances of the same class. In that case, the people (each instance of the 'person' class) will all have a different color. %[https://theaisummer.com/Semantic_Segmentation/]

It turns out that semantic segmentation has many different applications such as autonomous vehicles, human-computer interaction, robotics, and photo editing/creativity tools. For instance, semantic segmentation is very crucial in self-driving cars and robotics because it is important for the models to understand the context in the environment in which they’re operating. %[https://heartbeat.fritz.ai/a-2019-guide-to-semantic-segmentation-ca8242f5a7fc]

\vspace{5mm}
\begin{figure}[htb]
	\begin{center}
		\includegraphics*[width=13cm, keepaspectratio]{obr/semseg.jpg}
	\end{center}
	\caption{Segmentation of an urban road scene} %[https://theaisummer.com/Semantic_Segmentation/]
	\label{cocka}
\end{figure}


\chapter{Research and theory}
\label{research}

\section{Convolutional Neural Networks}

Convolutional Neural Networks (CNN) are very similar to Neural Networks from the previous chapter. They became widely used after Krizhevsky et al. [] won the ImageNet challenge with a CNN. One reason for the recent success of CNN is that they have fewer neurons. This has two advantages. Firstly, such networks are cheaper to train. Secondly, reducing the number of neurons regularises the network and reduces the risk of overfitting. CNN are trained with backpropagation as well as perceptrons.  

\subsection{CNN Layer Types}
The fundamental blocks we developed for learning regular Neural Networks still apply here. CNN architectures make the explicit assumption that the inputs are images (usually of the size MxNx3 for RGB). Typical CNN architecture consists of layers that, in addition to the already presented principles, allow it to exploit the spatial and colour information encoded in the image.

\subsubsection{Convolution Layers}

In CNN, layer parameters consist of a set of learnable filters. Each filter is small spatially but extends through the full depth of the input volume. For example, a typical filter in the first layer of a CNN with RGB inputs has size 5x5x3. During the forward pass, we convolve each filter across the width and height of the input volume and compute dot products between the entries of the filter and the input at any position. Intuitively, the network learns filters that activate when they see some type of visual features such as edges of certain orientation or a blotch of some colour in the first layer. Now, we have an entire set of filters in each CONV layer, and each of them will produce a separate 2-dimensional activation map (sometimes called feature map). Finally, we stack these activation maps along the depth dimension and produce the output volume that becomes an input for other layers. [https://cs231n.github.io/convolutional-networks/]

FIGURE AND MATHEMATICS

\subsubsection{Pooling Layers}

The function of pooling layers is to progressively reduce the spatial size of the layers in the network and thus reduce the number of parameters. [https://cs231n.github.io/convolutional-networks/] A neuron in a pooling layer takes the outputs of several neighbouring feature maps and summarises their outputs into a single number. Max-pooling units, for example, summarise the outputs of nearby feature maps (in a 2×2 square for instance) by taking the maximum over the feature-map outputs. There are no trainable parameters associated with the pooling layers, they compute the output from the inputs using a pre-defined prescription. [mehlig]

\subsection{Examples of CNN Architectures}

Most CNN architectures were developed for image classification. This is achieved by combining the properties of CNN and FCN (perceptrons). We see that in the deepest stage, the output of the network is followed by a standard multilayer perceptron with softmax output. The role of CNN here is only to encode the significant features of a particular image into a lower-level representation. The FCN then takes this output, literarly flattens the output tensor and learns to classify it.

There have been introduced various architectures, each of them having a different number of convolution layers, size of the filters, stride taken by the filters during convolution, etc. In practice, one rarely designs a CNN from scratch; instead, it is advisable to choose the currently best-performing network, usually one that performs best on the ImageNet challenge.

Here is a summary of the milestone architectures presented in recent years:

\begin{itemize}
	\item \textbf{AlexNet}
	
	The first work that popularized Convolutional Networks in Computer Vision was the AlexNet []. The Network had very similar architecture to LeNet [], but was deeper, bigger, and featured Convolutional Layers stacked on top of each other (previously it was common to only have a single CONV layer always immediately followed by a POOL layer).
	
	\item \textbf{GoogLeNet}
	
	The ILSVRC 2014 winner was a Convolutional Network from Szegedy et al. from Google. Its main contribution was the development of an Inception Module that dramatically reduced the number of parameters in the network (4M, compared to AlexNet with 60M). Additionally, this paper uses Average Pooling instead of Fully Connected layers at the top of the ConvNet, eliminating a large number of parameters that do not seem to matter much.
	
	\item \textbf{VGGNet}
	
	The runner-up in ILSVRC 2014 was the network from Karen Simonyan and Andrew Zisserman that became known as the VGGNet. Its main contribution was in showing that the depth of the network is a critical component for good performance. Their final best network contains 16 CONV/FC layers and, appealingly, features an extremely homogeneous architecture that only performs 3x3 convolutions and 2x2 pooling from the beginning to the end. 
	
	\item \textbf{ResNet}
	
  	Residual Network developed by Kaiming He et al. was the winner of ILSVRC 2015. It features special skip connections and heavy use of batch normalization. The architecture is also missing fully connected layers at the end of the network.
		 
\end{itemize}







\newpage
\section{Semantic segmentation}

In semantic segmentation, one assigns a class to each pixel of an input image, unlike in the classification task, where one classifies the entire image. This section presents the most successful methods involving neural networks and supervised learning. 

Segmentation has always been one of the most fundamental areas of computer vision. The classical approaches are mostly based on the standard signal processing theory and some of them can still be implemented and give satisfactory results. However, this applies only to a limited number of use cases, where the conditions are very close to ideal and where the robustness of the algorithm is not crucial. To give an example of classical methods, one can refer to thresholding, region growing and mean-shift segmentation \cite{coufal}. More advanced methods using machine learning classification have also been introduced, such as TextonBoost, TextonForest and Random Forest \cite{segnet} \cite{bayesian}. These algorithms have fallen out of favour due to the massive success of ANN.

\subsection{Encoder-decoder architecture}

In the previous chapter, the CNN architectures designed for image classification were presented. The size of the output layer of these networks is determined by the number of categories of classification because the CNN transfers to a FCN in the end. In semantic segmentation, however, one needs to get an image of the same resolution as the input image containing the information about a class of every pixel. To do this, the common scheme is introduced: the first part of the network is left unchanged but now, instead of the transition to FCN, various methods are implemented to upsample the encoded image features from the deepest layer of the CNN. This scheme is referred to as the encoder-decoder architecture. 

\vspace{4mm}
\begin{figure}[h]
	\begin{center}
		\includegraphics*[width=11cm, keepaspectratio]{obr/segnet.png}
	\end{center}
	\vspace{4mm}
	\caption{SegNet - an example of encoder-decoder CNN architecture. \cite{segnet}} 
	\label{encoder}
\end{figure}

The purpose of the encoder is to downsample the input images while still representing the significant features. The decoder part of the algorithm then upsamples the output of the encoder to the original input image size. This is usually done by performing reverse operations to max-pooling and convolution. The last part of the decoder typically gives the final segmented image. \\

Shortly after the success of CNN in image classification challenges, there have been introduced several segmentation architectures using CNN as the encoder. Some of the state-of-the-art were, for instance, FCN, DeconvNet and U-Net \cite{segnet}. These networks share the idea of having CNN incorporated as the encoder but differ in the form
of the decoder part. However, the problem of training such networks due to a large number of trainable parameters, the design of the decoder and hence the need of introducing the cumbersome multi-stage training made them very difficult to use in practice \cite{segnet}. SegNet \cite{segnet} introduced in 2015 differs from these architectures as it has a significantly lower number of parameters and the design od the encoder-decoder network allows it to be trained via standard method using backpropagation and SGD.

\subsubsection{Input upsampling}

The upsampling in the decoding part of the network is done via two mechanisms: learnable transposed convolution and unpooling. 

Transposed convolution, as well as the standard convolution used in CNN, uses learnable filters. The difference is that it takes a single input point instead of a region, uses it to multiply each element of the filter and creates its imprint in the output layer. This scheme is illustrated in Figure \ref{transposed} (left).

\vspace{4mm}
\begin{figure}[h]
	\begin{center}
		\includegraphics*[width=5cm, keepaspectratio]{obr/transposed.png}
	\end{center}
	\vspace{4mm}
	\caption{Transposed convolution. \cite{theano}} 
	\label{transposed}
\end{figure}

There are several ways to impelement unpooling. In an encoder-decoder architecture, the corresponding layers in the encoder and decoder can for example share the original locations of the elements that were pooled in the encoding part. The decoder then uses these indices for upsampling, as shown in Figure \ref{transposed} (right). This reconstructs the original positions of the features in the original image. Unpooling operation does not have any learnable parameters. 

\vspace{4mm}
\begin{figure}[h]
	\begin{center}
		\includegraphics*[width=11cm, keepaspectratio]{obr/unpool.png}
	\end{center}
	\vspace{4mm}
	\caption{Max-unpooling. The locations of the maximum elements were saved during max-pooling. The remaining elements are set to zero.} 
	\label{unpool}
\end{figure}

\newpage
\subsection{SegNet}

SegNet is a deep encoder-decoder architecture for multi-class semantic segmentation researched and developed by members of the Computer Vision and Robotics Group at the University of Cambridge. \cite{segnet_tut}

The architecture consists of a sequence of encoders and a corresponding set of decoders followed by a pixel-wise Softmax classifier. Typically, each encoder consists of one or more convolutional layers with batch normalisation and a ReLU non-linearity, followed by max-pooling. SegNet uses max-pooling indices in the decoders to perform upsampling of low-resolution activation maps (Figure \ref{encoder}). The entire architecture can be trained using stochastic gradient descent. \cite{segnet_tut}

\subsubsection{SegNet - encoder}

The architecture of the encoder network is topologically identical to the 13 convolutional layers in the VGG16 network. Each encoder in the encoder network performs convolution with a filter bank to produce a set of activation maps. These are then batch normalized. Then an element-wise ReLU is applied. Following that, max-pooling with a 2×2 window and stride 2 is performed. Storing the max-pooling indices, i.e, the locations of the maximum feature value in each pooling window is memorized for each encoder feature map. \cite{segnet}

\subsubsection{SegNet - decoder}

The decoders in the network upsample their input feature maps using the memorized max-pooling indices from the corresponding encoder feature maps. These feature maps are then convolved (using transposed convolution) with a trainable decoder filter bank to produce dense feature maps. A batch normalization step is then applied to each of these maps. The high dimensional feature representation at the output of the final decoder is fed to a trainable soft-max classifier. The predicted segmentation corresponds to the class with maximum probability at each pixel. \cite{segnet} The schematic of the SegNet architecture can be found in Attachment XY.

\subsection{Bayesian SegNet}

Bayesian SegNet is a probabilistic variant of SegNet. It can predict pixel-wise class labels together with a measure of model uncertainty.  This is achieved by Monte Carlo sampling with dropout at test time. The authors of the paper show that modelling uncertainty improves segmentation performance by 2-3 \% compared to SegNet. The schematic of the Bayesian SegNet architecture can be found in Attachment XY. \cite{bayesian}

\subsubsection{Monte Carlo Dropout}

Monte Carlo Dropout (MCDO) sampling allows to understand the model uncertainty of the result. As explained in Chapter \ref{dropout_sec}, the standard weight averaging dropout proposes to remove dropout at test time and scale the weights proportionally to the dropout percentage. MCDO, on the other hand, samples the network with randomly dropped out units at test time. \cite{bayesian}

It is important to highlight that the probability distribution from MCDO sampling is significantly different from the ‘probabilities’ obtained from a softmax classifier. The softmax function approximates relative probabilities between the class labels, but not an overall measure of the model’s uncertainty. \cite{bayesian}

\subsection{Evaluating segmentation performance}

The performance of semantic segmentation is often described by so called IoU (intercestion over union) metrics. IoU is the area of overlap between the predicted segmentation and the ground truth divided by the area of union between the predicted segmentation and the ground truth, as shown in the figure below. This metric ranges from 0–1 (0–100\%) with 0 signifying no overlap and 1 signifying perfectly overlapping segmentation.

\vspace{4mm}
\begin{figure}[h]
	\begin{center}
		\includegraphics*[width=9cm, keepaspectratio]{obr/iou.png}
	\end{center}
	\vspace{4mm}
	\caption{Intersection over union. \cite{iou}} 
	\label{iou}
\end{figure}





\chapter{Implementation and method}

In this chapter, the original Caffe [] implementation of SegNet and Bayesian SegNet together with their simplified version SegNet Basic and Bayesian SegNet Basic will be tested on a custom dataset. Part of this will be evaluating the effect of various training hyperparameters, solvers, data augmentation techniques etc. This will also give the instructions on how to set up the software/hardware environment to run Caffe framework.

\section{CPU vs. GPU for Training ANN}

CPU is the main processing unit of a computer. Current CPU's usually have 4 to 8 separate cores, which allows them to run several tasks in parallel. Graphics processing unit (GPU) was originally designed for performing only rendering computer graphics. Table XY gives and idea of how these two computational units differ in terms of the kind of task they're designed for. Note that CPU has much lower number of cores, but these run at high frequency and are very capable in terms of the instructions they perform. Therefore, CPU's are great for sequential tasks. On the other hand, GPU comprises of a large number of 'simple' cores, which makes it better for computing parallel tasks. 

In terms of the available memory, CPU doesn't have its own resources (apart from the very small memory sections called caches) and has access to the system's RAM, whose size is very often between 8 and 32 GB for powerful PC's. GPU's, on the other hand, have their own block of RAM on the chip because the access top the main system's RAM has usually many bottlenecks. The size of the RAM for the top-end GPU's ranges from 8 to 12 GB.

The main part of the computations in Neural Networks in general is matrix multiplication. For this, GPU has the power of performing these operations by parts in parallel which speeds up the training significantly. 

There have been created abstraction frameworks such as CUDA and OpenCL, that allow programmers do write their code in an usual manner and run it directly on GPU. For the purposes of Neral Networks, NVIDIA has developed a library of the most commonly used CUDA primitives named cuDNN. 

\subsubsection{Tensor Cores}

Tensor Core is a special GPU feature offered by NVIDIA cards. It enables mixed-precision computing, dynamically adapting calculations to accelerate throughput while preserving accuracy. The latest generation expands these speedups to a full range of workloads. From 10x speedups in AI training with Tensor Float 32 data type, to 2.5x boosts for high-performance computing with floating point 64 (double precision). [nvidia site]

\section{ANN frameworks}

As the architecture and training of Neural Networks are getting more complicated, there is a room for programmers to make ANN frameworks such as Caffe, TensorFlow, and PyTorch as user friendly as possible. The idea of these software tools is to make a higher abstraction of the architecture of the network called computational graph. The user can therefore think of designing and training the network separately by applying an optimizer to the computational graph that represents the layers of the network. 

Caffe is a deep learning framework made with expression, speed, and modularity in mind. It is developed by Berkeley AI Research (BAIR) and by community contributors. [berkeley caffe] The main difference from the other mentioned frameworks is that the user often doesn't need to write any code at all. The architecture of the network (the computational graph) is created in a .prototxt file, which is a standard text file in which one fills in the subsequent layers of the network in the desired order. Also, rather than having a optimizer object, one creates another .prototxt file that contains parameters such as the optimizer type (SGD, Adam, etc.), learning rate, momentum constant and others. After both of this files are created, the user runs Caffe computation from the command line. The core of the framework is written in C++. Pre-built binaries are called When the computation is started.

Caffe also has bindings for Python (CPW - Caffe Python Wrapper) and Matlab, which if very useful for evaluating the training statistics. 

\section{Setting up Environment for Caffe}

\subsection{Hardware configuration}

The GPU used for the computations has been picked according to the most up-to-date benchmarks and recommendations found online [source]. When choosing GPUs in general, one needs to decide between ATI/AMD and NVIDIA chips. For this case however, NVIDIA is the choice because it's way more 'ANN-friendly' as it's offering more features specifically designed for ANN computations. 

It is also advisable to use SSD in the PC configuration, because the data flow begins from reading the training data (images) from a storage, in this case from the computer's hard drive. Another way is moving the training data into RAM before the training is initiated [source, dalasi info]. 

Table XY shows the complete PC specifications used for training SegNet for the purposes of this thesis.  

\subsection{Software configuration} 

\subsubsection{Operating System} 

The standard platform for running Caffe is Ubuntu, which is a Linux distribution from Cannonical based on Debian. The environment used is Ubuntu 18.04 LTS 64 bit. Is is important to let the Ubuntu installer download the latest updates, or, after the installation, invoke the update command to ensure that the most up-to-date packages will be installed. For this, one can call

\begin{lstlisting}[language=bash]
$ sudo apt update
$ sudo apt upgrade
\end{lstlisting}

\subsubsection{Enabling NVIDIA driver}

Ubuntu 18.04 enables the default Nouveau graphics driver after installation. Before taking other steps, it is vital to disable the Nouveau driver and use NVIDIA instead in \textit{Application menu -> Software \& Updates -> Additional drivers
	-> Using NVIDIA driver metapackage from nvidia-driver-XYZ (proprietary, tested) -> Apply changes.} The driver version used is nvidia-driver-440.

[https://www.linuxbabe.com/ubuntu/install-nvidia-driver-ubuntu-18-04]

\subsubsection{CUDA installation}

CUDA version is determined by the version of cuDNN compatible with the used Caffe version, which is cuDNN 5.1 in this case. The corresponding CUDA version is CUDA 8.0. On Ubuntu 18.04, the procedure is the following:

\begin{itemize}
	\item \textbf{Download CUDA 8.0 runfile.} Go to \href{https://developer.nvidia.com/cuda-80-ga2-download-archive}{CUDA Legacy Releases} and look for 'CUDA Toolkit 8.0 GA2 (Feb 2017)'. The standard .deb installer support only Ubuntu 16.04 LTS and therefore the installation must be performed via the runfile method. Navigate to Linux -> x86\_64 -> Ubuntu -> 16.04 -> runfile (local) -> Base installer. Also, download the Patch file. 
	
	\item \textbf{Perform the runfile installation of CUDA.} Open the Ubuntu Terminal (Ctrl+Alt+T) and run
	
	\begin{lstlisting}[language=bash]
	$ cd /path/to/cuda_8.0.61_375.26_linux.run # Navigates to folder with CUDA
	$ sudo chmod a+x cuda*		# Makes the cuda*.run executable
	$ ./cuda*.run --tar mxvf 	# Unpacks the .runfile content
	$ sudo cp InstallUtils.pm /usr/lib/x86_64-linux-gnu/perl-base  # Copy one of the extracted files to perl-base
	$ sudo sh cuda_8.0.61_375.26_linux.run --override # Start the installation 
		# The licence agreement
		$ accept 
		# You are attempting to install on an unsupported configuration. Do you wish to continue?
		$ yes 
		# Install NVIDIA Accelerated Graphics Driver for Linux-x86_64 375.26?
		$ no
		# Install the CUDA 8.0 Toolkit?
		$ yes 
		$ <press enter> (leave deafult location)
		# Do you want to install a symbolic link at /usr/local/cuda?
		$ yes
		# Install the CUDA 8.0 Samples?
		$ no
	\end{lstlisting}
	
	After the installation is done, ignore the '***WARNING: Incomplete installation!' statement, because the NVIDIA driver is already installed. 
	
	Now run the CUDA 8.0 Patch 2 installation is a similar fashion:
	
	\begin{lstlisting}[language=bash]
	$ sudo sh cuda_8.0.61.2_linux.run
	\end{lstlisting}
	
	\item \textbf{Perform the post-installation actions.} The system needs to know the location of CUDA executables. The common way is to set these "PATH" variables in the current session of the Terminal. However, it's useful to add these permanently to '\textasciitilde{}/.bashrc' :
		
	\begin{lstlisting}[language=bash]
	$ sudo gedit ~/.bashrc # Opens the .bashrc file in text editor
	\end{lstlisting}
	
	In the text editor, append the following two statements to the end of the file:
	
	\begin{lstlisting}[language=bash]
	export PATH=/usr/local/cuda-8.0/bin${PATH:+:${PATH}}
	export LD_LIBRARY_PATH=/usr/local/cuda-8.0/lib64\
			${LD_LIBRARY_PATH:+:${LD_LIBRARY_PATH}}
	\end{lstlisting}	
	
	From this point, all newly opened Terminal sessions should have the paths set correctly. 
	
\end{itemize}

\subsubsection{cuDNN installation}

The NVIDIA CUDA Deep Neural Network library (cuDNN) is a GPU-accelerated library of primitives for deep neural networks. It provides highly tuned implementations for standard routines such as forward and backward convolution, pooling, normalization, and activation layers. [https://developer.nvidia.com/cudnn] 

\begin{itemize}
	\item \textbf{Download cuDNN 5.1 for CUDA 8.0.} To get the appropriate cuDNN version for Caffe and CUDA 8.0, go to \href{https://developer.nvidia.com/rdp/cudnn-archive}{cuDNN Archive} (requires login) and look for \textit{Download cuDNN v5.1 (Jan 20, 2017), for CUDA 8.0 -> cuDNN v5.1 Library for Linux}. Extract the archive, navigate to the extracted folder and copy the files to the CUDA 8.0 installation folder:
	
	\begin{lstlisting}[language=bash]
	$ tar -xf cudnn-8.0-linux-x64-v5.1.tgz 
	$ cd cuda
	$ sudo cp -a include/cudnn.h /usr/local/cuda/include/
	$ sudo cp -a lib64/libcudnn* /usr/local/cuda/lib64/
	\end{lstlisting}	
\end{itemize}

\subsubsection{Setting up Python Editor}

The scripts for evaluating SegNet performace are written in Python. It's advisable to use Pycharm Community Edition for an editor, because it offers a very convenient combination of GUI and the standard command line environment.

A good practice is to use Python virtual environment to easily maintain the required packages and to make the project transferable to another Linux PC. In Pycharm, we can do this in an active Pycharm project by going to \textit{File -> Settings -> Project -> Project Interpreter -> <wheel icon on the right> -> Add}. The standard choice is the Virtualenv Environment. The Base interpreter location on a fresh Ubuntu installation is '/usr/bin/python3.6'. When we click OK, Pycharm creates a 'venv' folder at the specified location including all package files we install.

When the virtualenv is configured properly, is will automatically be activated when we enter the Linux Terminal session by clicking on 'Terminal' located at the bottom bar of Pycharm. From this Terminal, we'll be launching all SegNet scripts and use it to install the required packages by calling 'pip3 install <package-name>'.

\subsection{Building Caffe for SegNet} 

The Caffe code is an open-source software. The authors of the SegNet created a slightly modified version of Caffe (caffe-segnet) that supports special SegNet layer types (upsample, bn, dense\_image\_data and softmax\_with\_loss (with class weighting)).

In addition, since the original caffe-segnet supports just cuDNN v2, which is not supported for new pascal based GPUs, there's another version of caffe-segnet from [TimmoSaemannGithub] that supports cuDNN 5.1 and decreases the inference time by 25 \% to 35 \%. This version has therefore been selected for running SegNet. From this point on, the term 'Caffe' will be equivalent to 'caffe-segnet' in the text.

\begin{itemize}
		
	\item \textbf{Install Caffe dependencies.} Caffe is available as a source code and therefore needs to be compiled on the target platform. For this, several steps need to be taken to ensure that all libraries are available during the build. 
	
	\begin{lstlisting}[language=bash]
	$ sudo apt install python3-opencv 			# OpenCV, version 3
	$ sudo apt-get install libatlas-base-dev 	# Atlas BLAS library
	$ sudo apt-get install libprotobuf-dev libleveldb-dev libsnappy-dev libopencv-dev libhdf5-serial-dev protobuf-compiler
	$ sudo apt-get install libboost-all-dev		# Boost
	$ sudo apt-get install libgflags-dev libgoogle-glog-dev liblmdb-dev
	$ sudo apt-get install python3-pip
	$ sudo pip3 install protobuf
	$ sudo apt-get install the python3-dev
	\end{lstlisting}
	
	\item \textbf{Download Caffe (caffe-segnet-cudnn5) source code.} Go to \href{https://github.com/TimoSaemann/caffe-segnet-cudnn5}{Timmoe Saemann's Github repository} and clone/download it. 
	\item \textbf{Set the build configuration file.} The build is done via the 'make' command, which needs the 'Makefile.config' file to be present in the parent directory ('caffe-segnet-cudnn5-master'). This file contains the build options and needs to be configured properly. Fortunately, the correct form of 'Makefile.config' is part of this thesis and can be found in the Attachment XY. 
	
	\item \textbf{Install gcc/g++ compliers.} The CUDA/cuDNN libraries used during the build are compatible only with gcc/g++ compilers of version 5. To install these, run:
	
	\begin{lstlisting}[language=bash]
	$ sudo apt install gcc-5 g++-5
	# Create symbolic links so CUDA can see the proper compiler binaries
	$ sudo ln -s /usr/bin/gcc-5 /usr/local/cuda/bin/gcc
	$ sudo ln -s /usr/bin/g++-5 /usr/local/cuda/bin/g++
	\end{lstlisting}
	
	\item \textbf{Start the build.} Once the 'Makefile.config' file is located in the 'caffe-segnet-cudnn5-master', everything should be ready for the final step. Type these commands to initiate and test the Caffe build (don't forget to build pycaffe (Caffe Python Wrapper)):
	
	\begin{lstlisting}[language=bash]
	make all -j4	# start build
	make test -j4	# test build
	make runtest	# run Caffe and test it
	make pycaffe	# build pycaffe 
	\end{lstlisting} 	
\end{itemize}
[https://mc.ai/installing-caffe-on-ubuntu-18-04-with-cuda-and-cudnn/]
\newpage
\section{Image annotation}

In supervised learning, one needs to manually create the training data consisting of inputs and corresponding targets (called ground truths in segmentation). There's a variety of annotation tools available on the internet, both under commercial and free licenses. 

\subsubsection{Labelbox}

Labelbox \cite{labelbox} is a paid online annotation tool. The best feature of Labelbox is that it allows sharing the datasets with other users and therefore speeding up the labeling significantly. Labelbox offers free access to students to the full version. When the labeling is finished, one exports the image/label pairs to a \textit{.JSON} file. This file contains links to the annotated images that are stored online and it is necessary to download them separately (Labelbox is still in development, this is valid at the time of publishing). To automate this process, one can call the \textit{download()} function from \textit{utilities.py} which is available in \cite{filip_github}. \\

The final train, validation and test datasets used contain 2600 + 90 + 179 images from an outdoor environment. The location where the images were taken is the same in all cases. However, the scenes differ in their weather conditions, daytime, type of path, etc. The corresponding \textit{.JSON} files are available at \cite{filip_github}.

%\section{Caffe Commands}

%The documentation for Caffe is not one of the best and sometimes it might be quite %tricky to find reasonable answers. Therefore, this section will give a brief %summary of the most important terms and parameters in Caffe library.

\section{Setting up SegNet}

Caffe implementation of ANN typically consists of four \textit{.prototxt} files: \textit{train.prototxt}, \textit{solver.prototxt}, \textit{test.prototxt} and \textit{inference.prototxt}. The \textit{train}, \textit{test} and \textit{inference} files are almost identical except for a few differences in the very first/last layers of the network. The \textit{train} file is used together with the \textit{solver} file to train the network. The network architecture is determined by the \textit{train} file and the parameters for optimization reside in the \textit{solver} file. The \textit{test} file is used by Caffe when one needs to test the network periodically during training on a validation dataset. \cite{caffe} The \textit{inference} file is used for running the trained network. The files used in this section are available at \cite{filip_github}

\subsection{Solver settings}

The \textit{solver} file contains the optimization parameters. The description of the parameters can be found in the original Caffe documentation \cite{caffe}. An example of the parameters used can be found in the snippet below. 

\begin{lstlisting}[caption={Contents of \textit{solver.prototxt} \cite{filip_github}},captionpos=b]
// Training file
net: "/path/to/train.prototxt"	
// Caffe GPU version
solver_mode: GPU
// Solver type		
type: "AdaDelta"
// Initial learning rate, changes according to lr_policy		
base_lr: 0.061		
// Determines how the learning rate changes during training
lr_policy: "fixed"	
// Show loss and accuracy every 'display' iterations
display: 130
// Max number of iteration. One iteration = a pass of one mini batch			
max_iter: 3000	
// Weight decay factor		
weight_decay: 0.0005
// Saves the weights after 'snapshot' iterations
snapshot: 1000000		
snapshot_prefix: "/path/to/snap" 
\end{lstlisting}

\subsection{Training}

\subsubsection{Input layer and input pre-processing}

The \textit{train} file begins with the \textit{DenseImageData} layer. This layer specifies the size of the mini-batch. The value is limited by the amount of memory that the GPU offers. When a larger size of the mini batch is needed, Caffe can specify the \textit{iter\_size} parameter in the \textit{solver} file. The total mini-batch size in Caffe is always a result of $iter\_size \cdot batch\_size$. By default, the value of \textit{iter\_size} is set to 1. \cite{caffe}

The \textit{shuffle} parameter in the \textit{DenseImageData} layer determines whether the training dataset is shuffled after each epoch. This is usually desirable as it helps the optimization algorithm by adding more stochasticity to the computation. The \textit{mirror} parameter applies random mirrors to the input data and hence augments the dataset. If one needs to apply more complex data augmentation techniques, it is necessary to perform them separately and feed the \textit{DenseImageData} layer with already processed images. \cite{caffe}

\begin{lstlisting}[caption={Input layer in \textit{train.prototxt} \cite{filip_github}},captionpos=b]
name: "segnet_train"
layer {
name: "data"
type: "DenseImageData"
top: "data"
top: "label"
dense_image_data_param {
	source: "/path/to/train_image_paths.txt"
	batch_size: 4   			    			
	shuffle: true
	mirror: true	
	}
// Per-channel mean
transform_param {
	mean_value: 129		// B component
	mean_value: 126		// G
	mean_value: 126		// R
	}
}
\end{lstlisting} 

Images and labels are loaded as \textit{.jpg} and \textit{.png} files directly from the hard drive (there are more methods that Caffe offers, see \cite{caffe}). The path to the \textit{image\_paths.txt} file that contains the image/label paths in the following format

$$
\text{\textit{/path/to/image.jpg /path/to/label.png}}
$$

\noindent is entered as the \textit{source} parameter of the \textit{DenseImageData} layer. This file is generated using the \textit{make\_txt()} function from \textit{utilities.py}. The script will also make separate directories for training, testing and validation datasets by calling \textit{make\_dirs()}.

The method used for the mean subtraction was the per-channel mean. The \textit{per\_channel\_mean} function in \textit{utilities.py} calculates the mean values for R, G and B components of the images in the training set. These three numbers are then placed into the \textit{DenseImageData} layer in BGR order.

\subsubsection{Output dimensions}

In the original version, SegNet has 11 segmentation classes. This corresponds to the pixel values in the \textit{.png} label files starting from zero. For instance, the segmentation mask for the class number 1 has a pixel value of 0 in the label file, etc. However, the goal of this thesis is to set the network to segment only two classes - \textit{path, background}. To change the size of the output classifier, it is necessary to change the output dimensions of the last \textit{conv} layer:

\begin{lstlisting}[caption={Setting number of outputs in \textit{train.prototxt} \cite{filip_github}},captionpos=b]

// The last conv layer in the network
layer {
	bottom: "conv1_2_D"
	top: "conv1_1_D"
	name: "conv1_1_D"
	type: "Convolution"
	.
	.
	.
	convolution_param {
		.
		.
		.
		num_output: 2		// Set this to the number of classes
		pad: 1
		kernel_size: 3
	}
}
\end{lstlisting}

\newpage
\subsubsection{Softmax classifier}

\enquote{\textit{When there is large variation in the number of
	pixels in each class in the training set (e.g road, sky and building
	pixels dominate the CamVid dataset) then there is a need to weight
	the loss differently based on the true class. This is termed class
	balancing. We use median frequency balancing [13] where the
	weight assigned to a class in the loss function is the ratio of the
	median of class frequencies computed on the entire training set
	divided by the class frequency. This implies that larger classes in
	the training set have a weight smaller than 1 and the weights
	of the smallest classes are the highest. We also experimented
	with training the different variants without class balancing or
	equivalently using natural frequency balancing.}} \cite{segnet}

\begin{lstlisting}[caption={Output layers of \textit{train.prototxt} \cite{filip_github}},captionpos=b]
// The softmax classifier with cross-entropy loss
layer {
	name: "loss"
	type: "SoftmaxWithLoss"
	bottom: "conv1_1_D"
	bottom: "label"
	top: "loss"
	softmax_param {engine: CAFFE}
	loss_param: {
		weight_by_label_freqs: false	     
	}
}
layer {
	name: "accuracy"
	type: "Accuracy"
	bottom: "conv1_1_D"
	bottom: "label"
	top: "accuracy"
	top: "per_class_accuracy"
}
\end{lstlisting}

%% Snippet XY, the last two layers of the train network

SegNet uses the cross-entropy loss as the loss function for training the network. In Caffe, median frequency balancing is available via the \textit{weight\_by\_label\_freqs} parameter of the \textit{SoftmaxWithLoss} layer. Since the dataset used has only two classes whose occurrences can be considered balanced, this option is set to \textit{false}. 

\subsubsection{Training initialization}

\noindent Training the network from scratch is initiated by entering these commands:

\begin{lstlisting}[language=bash]
# Navigate to the caffe-segnet folder
$ cd /path/to/caffe-segnet/build/tools/
# Initiate training from scratch
$ ./caffe train -solver /path/to/segnet_solver.prototxt
# or resume training from a solver checkpoint (snapshot)
$ ./caffe train -solver /path/to/segnet_solver.prototxt -snapshot /path/to/snapshot_iter_XY.solverstate
\end{lstlisting}

The encoder and decoder weights are initialized using the MSRA method by default. Another scenario is when we want to use transfer learning (see Caffe Model Zoo in \cite{caffe} where people share their weights and networks). In this case, Caffe needs a path to the \textit{.caffemodel} file of the pre-trained network. The corresponding command would be:

\begin{lstlisting}[language=bash]
$ ./caffe train -solver /path/to/solver.prototxt -weights /path/to/pre_trained_weights.caffemodel
\end{lstlisting}

There are multiple ways of tuning the pre-trained model when using transfer learning. For instance, one can experiment with the learning rate of the pre-trained weights: they can either stay unchanged (zero learning rate) or the learning rate applied to them is lower than the global value used in other layers. \cite{stanford-github} In encoder-decoder architecture, one usually applies transfer learning only to the encoder network as it has no other purpose than extracting general features from the image. The corresponding setting in the \textit{train} file is the set of \textit{lr\_mult} parameters by which the learning rate for the layer is multiplied. An example of setting a Caffe layer where that layer stays unchanged can be found in the snippet below.

\begin{lstlisting}[caption={Setting up \textit{train.prototxt} for transfer learning \cite{filip_github}},captionpos=b]
layer {
	bottom: "data"
	top: "conv1_1"
	name: "conv1_1"
	type: "Convolution"
	// Learning rate factor - weights
	param {
		lr_mult: 0			// Zero value corresponds to freezing this parameter
		decay_mult: 0		// Zero value corresponds to freezing this parameter
	}
	// Learning rate factor - thresholds
	param {
		lr_mult: 0			// Zero value corresponds to freezing this parameter
		decay_mult: 0		// Zero originally, remains unchanged
	}
	.
	.
	.
}
\end{lstlisting}

% [https://arxiv.org/pdf/1411.4734.pdf]

\newpage
\subsection{Inference}

The network is ready to be deployed in this phase. At this point, it is very convenient to use pycaffe for running the model by feeding it with input data and calculating the segmentation accuracy. To run the segmentation, several preparation steps must be taken first.

\subsubsection{Calculating statistics for batch normalisation}
The batch normalisation layers in SegNet shift the input feature maps according to their mean and variance statistics for each mini- batch during training \cite{mehlig}. At inference time, we must use the statistics for the entire dataset and obtain the final \textit{.caffemodel} for the inference phase. \cite{segnet_get_started} We do this by calling \textit{compute\_bn\_statistics.py} which is meant to be run from the command line and needs to get command-line parameters. In PyCharm, we need to switch to Virtual Environment (venv) by opening Terminal and call:

\begin{lstlisting}[language=bash]
(venv) user@user:/path/to/Scripts$ python3 original_compute_bn_statistics.py /path/to/train.prototxt /path/to/snap_iter_XY.caffemodel /path/to/inference_folder
\end{lstlisting}

The network architecture for the inference is now in the \textit{inference} file and the same is in the \textit{train} file apart from the input and output layers and the settings of the batch normalisation layers. The snippet below shows the changes of the output: the loss function is no longer computed and the only output we care about is the set of softmax probabilities. The \textit{DenseImageData} layer is also skipped, because the data will be provided via pycaffe. Part of this is switching all batch normalisation layers to the INFERENCE mode. \cite{issue}

The script takes the desired \textit{.caffemodel} file specified in \textit{snap\_iter\_XY.caffemodel}, calculates new $ \gamma, \beta $ parameters for the batch normalisation layers and saves everything to \textit{final\_weights.caffemodel}. The new \textit{.caffemodel} file is now stored in the specified \textit{inference\_folder}. \cite{issue}

\begin{lstlisting}[caption={Replacing input layer type in \textit{inference.prototxt} \cite{filip_github}},captionpos=b]
// Inference, input layer
name: "segnet_inference"
input: "data"
input_dim: 1	// Always 1 for SegNet
input_dim: 3
input_dim: 360
input_dim: 480
\end{lstlisting}

\subsubsection{Running the segmentation}

The script \textit{segnet\_inference.py} is used for running the segmentation. One must provide the network with images either by specifying a path to a video file or by specifying a sequence of image names to look for in the image folder (this is a standard OpenCV convention). In each step of the algorithm, we must subtract the per-channel mean from the input image that is being processed. This is part of the script and one only needs to provide the BGR values used at train time.

Once an appropriate test set of images is ready, the segmentation is started by calling:

\begin{lstlisting}[language=bash]
(venv) user@user:/path/to/Scripts$ python3 segnet_inference.py /path/to/inference.prototxt /path/to/final_weights.caffemodel /path/to/videofile.avi 
\end{lstlisting}

\subsection{Testing}

The \textit{test} file is used only for calculating the loss of the validation dataset. It is very similar to the \textit{train} file: it has a \textit{DenseImageData} layer with a path to the validation dataset, \textit{mirror} and \textit{shuffle} parameters set to false, \textit{batch\_size} to 1 and the \textit{SoftmaxWithLoss} followed by \textit{Accuracy} layers as the output. The subtraction of the per-channel mean is still present and the values computed from the training dataset are the same as in the training phase. 

For testing, it is necessary to use the \textit{.caffemodel} file generated by \textit{compute\_bn\_statistics.py} to ensure the proper function of the batch normalisation layers, which must be in the INFERENCE mode and must differ from the settings of the \textit{train} file.

\begin{lstlisting}[caption={Setting up the input layer of \textit{test.prototxt} \cite{filip_github}},captionpos=b]
name: "segnet_test"
layer {
	name: "data"
	type: "DenseImageData"
	top: "data"
	top: "label"
	dense_image_data_param {
		source: "/media/phil/SegNet/data/custom/val_linux.txt"	
		batch_size: 1		// Always 1 for SegNet
	}
	// Per-channel mean, BGR
	transform_param {
		mean_value: 129
		mean_value: 126
		mean_value: 126 
	}  
	
\end{lstlisting}

Testing is executed similarly as training using the command line:

\begin{lstlisting}[language=bash]
# Navigate to the caffe-segnet folder
$ cd /path/to/caffe-segnet/build/tools/
# Initiate testing
$ ./caffe train -model /path/to/segnet_test.prototxt -weights /path/to/final_weights.caffemodel
\end{lstlisting}

\newpage
\subsection{Bayesian SegNet}

Since Bayesian SegNet differs from SegNet only in terms of added dropout layers and a different method of performing the inference the above-mentioned procedures for setting the solver and training are also applicable. Therefore, one can start the training by using commands from the previous section. One must also not forget to replace the paths of the \textit{train} and \textit{solver} files. 

The input layer in the \textit{inference} file has one major difference: unlike in SegNet, the first \textit{input\_dim} parameter at the top of the \textit{inference} file represents the number of MCDO samples and can be adjusted. At inference time, the script passes the same image \textit{input\_dim} times and simply averages the output of the network. For this reason, the dropout layers that are inactive by default when Caffe is performing inference (TEST, in Caffe terminology) must be set to active in this case. The corresponding parameter in the dropout layer is \textit{sample\_weights\_test: true}. 

The batch normalisation layers are set to INFERENCE mode. The final \textit{.caffemodel} is obtained the same way as in SegNet by calling \textit{compute\_bn\_statistics.py}. Here, unlike during inference time, the network's output is computed using the weight averaging technique instead of MCDO.

\begin{lstlisting}[caption={Setting MCDO in \textit{inference.prototxt} \cite{filip_github}},captionpos=b]
layer {
	bottom: "conv1_1"
	top: "conv1_1"
	name: "conv1_1_bn"
	type: "BN"
	bn_param {
		bn_mode: INFERENCE			// Inference mode of this batch norm. layer
		.
		.
		.
	}
}
.
.
.
layer {
	name: "encdrop5"
	type: "Dropout"
	bottom: "pool5"
	top: "pool5"
	dropout_param {
		dropout_ratio: 0.5
		sample_weights_test: true	// For Monte Carlo Dropout
	}
}
\end{lstlisting}

\newpage
The setting of the \textit{test} file remains the same as in SegNet: input is provided by the \textit{DenseImageData} layer, \textit{batch\_size} is set to 1 and the batch normalisation layers are in INFERENCE mode. The dropout layers can also be set to active here. This \textit{test} file still only serves for checking the validation loss.

The inference is initiated by calling:

\begin{lstlisting}[language=bash]
(venv) user@user:/path/to/Scripts$ python3 bayesian_segnet_inference.py /path/to/inference.prototxt /path/to/final_weights.caffemodel /path/to/videofile.avi 
\end{lstlisting}

Here the scripts also visualizes the statistics of MCDO sampling: the variance of the output segmentation computed from all MCDO samples.

\subsection{SegNet Basic and Bayesian SegNet Basic}

SegNet Basic and Bayesian SegNet Basic are networks provided by the SegNet authors and are similar to their full versions but have fewer layers (see Attachment 1). These shallow versions are used in the same way as their parent architectures. Therefore, the same training and inference procedures apply to SegNet+SegNet Basic and Bayesian SegNet+Bayesian SegNet Basic.

\newpage
\section{Optimization of hyperparameters}

Hyperparameters are parameters that are set before the training begins and do not change during the training. The choice of hyperparameters is a task in its own right and requires a sufficient amount of trial and error. There are some general approaches (mostly empirical) one can follow to find the right parameters. The goal is to ensure that the network reaches an optimal value of the loss function. \cite{stanford-github}

\subsubsection{Optimizer}

Every training of a neural network starts with the choice of an optimizer. As the most recent research suggests, Adam is the default choice for training CNNs. If the CNN is built from scratch, it is advisable to start from the simplest SGD optimizer and observe the values of the loss function to detect potential problems in the architecture or the code. \cite{stanford-L7}

\subsubsection{Learning rate}

The parameter that has the biggest effect on training is the learning rate: it is the first parameter one should set. It is recommended to start a coarse search first while observing the loss for both training and validation datasets for a few initial epochs. Then, after the training is done, choose a thinner interval of optimal learning rates and perform a finer search. \cite{stanford-L6}

As the learning rate has a multiplicative effect on the gradient accumulation during mini-batch training, it is logical to pick the values from the logarithmic space. \cite{stanford-L6}

\subsubsection{Cross-validation strategy}

This strategy is also referred to as early stopping. The idea is that one observes both training and validation loss during training. When these losses go apart, the network tends to overfit to the training data. This is a crucial step when finding optimal hyperparameters and it must always be checked. \cite{stanford-github}

\subsubsection{Regularisation}

When building a network from scratch, one starts with a simple SGD algorithm with no regularisation involved to ensure that the loss values are reasonable. After we check for errors in the code and after the network trains with SGD, regularisation is turned on. It is usually set to a very small value, typically of the order $ 10^{-4} $ \cite{stanford-L6}.




\section{Setting up Bayesian SegNet}

Since Bayesian SegNet differs from SegNet only in terms of added dropout layers and slightly different method of performing network inference, the above-mentioned procedures for setting the solver and training are applicable in the same way. One can therefore start the training by using the commands from the previous section and only replace the file paths of the TRAIN and SOLVER files. The TEST and INFERENCE files have only one major difference in the input layers. 

Unlike in SegNet, the \textit{batch\_size} parameter in the DenseImageData layer of the TEST file now represents the number of Monte Carlo Dropout samples used for output averaging, as described in Section XY. The same corresponds to the first dimension of the input layer in the INFERENCE file.

After calling the script \textit{compute\_bn\_statistics.py} on a trained Bayesian SegNet, we can start the inference by executing:

\begin{lstlisting}[language=bash]
(venv) user@user:/path/to/Scripts$ python3 bayesian_segnet_inference.py /path/to/inference.prototxt /path/to/final_weights.caffemodel /path/to/videofile.avi 
\end{lstlisting}

Here the scripts also visualizes the statistics of Monte Carlo sampling: the uncertainty and variance of the output segmentation.

\section{SegNet Basic and Bayesian SegNet Basic}

These shallow versions of SegNet and BayesianSegNet are used in the same way as their full versions above. The same procedures apply to SegNet/SegNet Basic and Bayesian SegNet/Bayesian SegNet Basic.
\chapter{Results}

ODECISTP PER CHANNEL MEAN PRO INFERENCI

bayesian takes longer to train

TRAINING STRATEGIES

TRANSFER LEARNING + BATCH NORMALIZATION

--------------TRAINING CAFFE NOTES---------------

We perform local contrast normalization [54] to the RGB input. 
We train the model with dropout and sample the posterior distribution over the weights at test time using dropout
to obtain the posterior distribution of softmax class probabilities. We take the mean of these samples for our segmentation prediction and use the variance to output model
uncertainty for each class. We take the mean of the per class
variance measurements as an overall measure of model uncertainty. We also explored using the variation ratio as
a measure of uncertainty (i.e. the percentage of samples
which agree with the class prediction) however we found
this to qualitatively produce a more binary measure of
model uncertainty. Fig. 2 shows a schematic of the segmentation prediction and model uncertainty estimate process.
\clearpage
\chapter{Conclusion and future work}

This thesis presented some of the most recent ANN architectures used for image segmentation. In the implementation part, the extensive step-by-step procedure for setting up both the software and hardware environments for running the Caffe implementation of the networks was described and tested on a fresh installation of the operating system. Part of this was showing the benefits of using Debian based distributions of Linux for working with libraries for ANN where the procedure described by shell commands is very clear and can be easily repeated on a different machine.

The Caffe implementation and auxiliary Python scripts for the presented networks were tuned for the purpose of this thesis which is performing segmentation on a custom dataset with two object classes. The dataset was created using the best currently available online annotation tool. In the training phase, the network architecture was adapted for various transfer learning strategies and showed the power of using pre-trained encoders when the dataset is small. 

During the inference phase, the time performance of the architectures was observed and compared.  


%
%\chapter{Závěr}
Závěr je opravdu nutný  rozhodně delší než jeden řádek. Závěr je opravdu nutný  rozhodně delší než jeden řádek. Závěr je opravdu nutný  rozhodně delší než jeden řádek. Závěr je opravdu nutný  rozhodně delší než jeden řádek. Závěr je opravdu nutný  rozhodně delší než jeden řádek. Závěr je opravdu nutný  rozhodně delší než jeden řádek. Závěr je opravdu nutný  rozhodně delší než jeden řádek. 
% nutné
\begin{thebibliography}{99}

\bibitem{mwiti}{MWITI, Derrick. 2019. A 2019 Guide to Semantic Segmentation. In: \textit{Heartbeat} [online]. Fritz AI. Available at: https://heartbeat.fritz.ai/a-2019-guide-to-semantic-segmentation-ca8242f5a7fc}

\bibitem{sergios}{KARAGIANNAKOS, Sergios. 2019. Semantic Segmentation in the era of Neural Networks. In: \textit{AI SUMMER} [online]. Available at: https://theaisummer.com/Semantic\_Segmentation/}

\bibitem{mehlig}{MEHLIG, Bernhard. 2019. Artificial Neural Networks. ArXiv.org [online]. Available at: https://arxiv.org/abs/1901.05639}

\bibitem{santiago}{PUENTE, Santiago. 2018. \textit{Single and Multi-Label Environmental Sound Classification Using Convolutional Neural Networks} [online]. Gothenburg. Available at: https://odr.chalmers.se/handle/20.500.12380/255604. Master's thesis. Chalmers University of Technology.}

\bibitem{goodfellow}{GOODFELLOW, Ian, Yoshua BENGIO a Aaron COURVILLE. \textit{Deep learning}. Cambridge, Massachusetts: The MIT Press, 2016. ISBN 978-026-2035-613.}

\bibitem{groman}{GROMAN, Martin. Tvorba umělé neuronové sítě pro výpočet termodynamických veličin [online]. Brno, 2019 [cit. 2020-06-07]. Available at: \url{http://hdl.handle.net/11012/175381}. Master's thesis. Vysoké učení technické v Brně. Fakulta strojního inženýrství. Ústav matematiky. Supervisor Tomáš Mauder.}

\bibitem{stanford-github}{CS231n: Convolutional Neural Networks for Visual Recognition: Lecture Notes. \textit{CS231n: Convolutional Neural Networks for Visual Recognition} [online]. Stanford: Stanford University. Available at: https://cs231n.github.io/}

\bibitem{stanford-L4}{Lecture 4|Introduction to Neural Networks \textit{YouTube} [online]. 11. August 2018. Available at: \url{https://www.youtube.com/watch?v=d14TUNcbn1k&list=PL3FW7Lu3i5JvHM8ljYj-zLfQRF3EO8sYv&index=4} }

\bibitem{stanford-L6}{Lecture 6|Training Neural Networks I \textit{YouTube} [online]. 11. August 2018. Available at: \url{https://www.youtube.com/watch?v=wEoyxE0GP2M&list=PL3FW7Lu3i5JvHM8ljYj-zLfQRF3EO8sYv&index=6} }

\bibitem{stanford-L7}{Lecture 7|Training Neural Networks II \textit{YouTube} [online]. 11. August 2018. Available at: \url{https://www.youtube.com/watch?v=_JB0AO7QxSA&list=PL3FW7Lu3i5JvHM8ljYj-zLfQRF3EO8sYv&index=7} }

\bibitem{coors}{COORS, Benjamin. 2016. \textit{Navigation of Mobile Robots in Human Environments with Deep Reinforcement Learning} [online]. Stockholm. Available at: \url{http://www.diva-portal.org/smash/record.jsf?pid=diva2\%3A967644&dswid=9005. Degree project. KTH Royal Institute of Technology.}}
		
\bibitem{eniola}{ALESE, Eniola. 2018. The curious case of the vanishing and exploding gradient. In: \textit{Medium} [online]. Available at: https://medium.com/learn-love-ai/the-curious-case-of-the-vanishing-exploding-gradient-bf58ec6822eb}

\bibitem{bushaev}{BUSHAEV, Vitaly. 2018. Adam — latest trends in deep learning optimization. In: Towards Data Science [online]. Towards Data Science. Available at: \url{https://towardsdatascience.com/adam-latest-trends-in-deep-learning-optimization-6be9a291375c}}

\bibitem{issue}{Batch Normalization Issue in SegNet. 2017. In: \textit{GitHub} [online]. GitHub. Available at: https://github.com/alexgkendall/caffe-segnet/issues/109}

\bibitem{arvi}{JONNARTH, Arvi. 2018. \textit{Camera-Based Friction Estimation with Deep Convolutional Neural Networks} [online]. Uppsala. Available at: https://pdfs.semanticscholar.org/4c35/becacb2aab803468eb38f19d8418d79c7c08.pdf. Master's thesis. Uppsala Universitet.}

\bibitem{krizhevsky}{KRIZHEVSKY, Alex, Ilya SUTSKEVER and Geoffrey HINTON. 2017. ImageNet classification with deep convolutional neural networks. \textit{Communications of the ACM} [online]. ACM. Available at: \url{http://web.b.ebscohost.com.ezproxy.lib.vutbr.cz/ehost/detail/detail?vid=0&sid=dcac7028-11f2-41e4-ba7d-d517a5a51a6f%40sessionmgr101&bdata=Jmxhbmc9Y3Mmc2l0ZT1laG9zdC1saXZl#AN=123446102&db=bth}}
		
\bibitem{lecun}{LECUN, Y, L BOTTOU, Y BENGIO and P HAFFNER. 1998. Gradient-based learning applied to document recognition. \textit{Proceedings of the IEEE} [online]. IEEE, 86(11), 2278-2324. Available at: \url{https://ieeexplore-ieee-org.ezproxy.lib.vutbr.cz/document/726791}}

\bibitem{szegedy}{SZEGEDY, Christian, Wei LIU, Yangqing JIA, Pierre SERMANET, Scott REED, Dragomir ANGUELOV, Vincent VANHOUCKE and Andrew RABINOVICH. 2014. Going Deeper with Convolutions. \textit{ArXiv.org} [online]. Ithaca: Cornell University Library, arXiv.org. Available at: \url{http://search.proquest.com/docview/2084489417/}}

\bibitem{vgg}{SIMONYAN, Karen and Andrew ZISSERMAN. 2014. Very Deep Convolutional Networks for Large-Scale Visual Recognition. \textit{Visual Geometry Group} [online]. Oxford: University of Oxford. Available at: \url{http://www.robots.ox.ac.uk/~vgg/research/very_deep/}}

\bibitem{resnet}{HE, Kaiming, Xiangyu ZHANG, Shaoqing REN and Jian SUN. 2015. Deep Residual Learning for Image Recognition. \textit{ArXiv.org} [online]. Ithaca: Cornell University Library, arXiv.org. Available at: \url{http://search.proquest.com/docview/2083823373}}

\bibitem{coufal}{COUFAL, J. Detekce cesty pro mobilní robot analýzou obrazu. Brno: Vysoké učení
	technické v Brně, Fakulta strojního inženýrství, 2010. 49 s. Master's thesis. Supervisor: Ing.
	Jiří Krejsa, Ph.D}

\bibitem{bayesian}{KENDALL, Alex, Vijay BADRINARAYANAN and Roberto CIPOLLA. 2016. Bayesian SegNet: Model Uncertainty in Deep Convolutional Encoder-Decoder Architectures for Scene Understanding. \textit{ArXiv.org} [online]. Ithaca: Cornell University Library, arXiv.org. Available at: \url{https://arxiv.org/abs/1511.02680}}

\bibitem{segnet}{BADRINARAYANAN, Vijay, Alex KENDALL and Roberto CIPOLLA. 2016. SegNet: A Deep Convolutional Encoder-Decoder Architecture for Image Segmentation.\textit{ ArXiv.org} [online]. Ithaca: Cornell University Library, arXiv.org. Available at: \url{https://arxiv.org/abs/1511.00561}}

\bibitem{segnet_tut}{KENDALL, Alex, Vijay BADRINARAYANAN and Roberto CIPOLLA. 2015. SegNet. \textit{Machine Intelligence Laboratory} [online]. Cambridge: University of Cambridge. Available at: \url{https://mi.eng.cam.ac.uk/projects/segnet/}}

\bibitem{zeltner}{ZELTNER, Felix. 2016. Autonomous Terrain Classification Through Unsupervised Learning [online]. Luleå. Available at: \url{http://ltu.diva-portal.org/smash/record.jsf?pid=diva2%3A1051763&dswid=-6301. Degree project. Luleå University of Technology.}}
	
\bibitem{segnet_get_started}{KENDALL, Alex. 2015. Getting Started with SegNet. \textit{Machine Intelligence Laboratory} [online]. Cambridge: University of Cambridge. Available at: \url{http://mi.eng.cam.ac.uk/projects/segnet/tutorial.html}}

\bibitem{iou}{ROSEBROCK, Adrian. 2016. Intersection over Union (IoU) for object detection. In: \textit{Pyimagesearch} [online]. pyimagesearch. Available at: \url{https://www.pyimagesearch.com/2016/11/07/intersection-over-union-iou-for-object-detection/}}

\bibitem{filip_github}{SegNet-Tutorial. 2020. \textit{GitHub} [online]. GitHub. Available at: \url{https://github.com/filipovyfusky/SegNet-Tutorial}}

\bibitem{nvidia}{\textit{NVIDIA} [online]. 2020. USA: NVIDIA. Available at: https://www.nvidia.com/}

\bibitem{nvidia_dev}{\textit{NVIDIA Developer} [online]. 2020. USA: NVIDIA. Available at: https://developer.nvidia.com/}

\bibitem{caffe}{\textit{Caffe} [online]. Berkeley: Berkeley AI Reseach. Available at: https://caffe.berkeleyvision.org/}

\bibitem{filip_github_caffe}{caffe-segnet-cudnn5. 2020. \textit{GitHub} [online]. GitHub. Available at: \url{https://github.com/filipovyfusky/caffe-segnet-cudnn5}}

\bibitem{aizawan_github}{SegNet implementation in Tensorflow. 2020. \textit{GitHub} [online]. GitHub. Available at: \url{https://github.com/aizawan/segnet}}

\bibitem{labelbox}{\textit{Labelbox} [online]. 2020. Available at: https://labelbox.com}

\bibitem{theano}{Convolution arithmetic tutorial. 2018. \textit{Deep Learning} [online]. LISA lab. Available at: \url{http://deeplearning.net/software/theano_versions/dev/tutorial/conv_arithmetic.html}}

\bibitem{gigabyte}{\textit{GIGABYTE} [online]. 2020. GIGABYTE. Available at: https://www.gigabyte.com/}

\end{thebibliography}
% nutné
\chapter*{List of Abbreviations}
\addcontentsline{toc}{chapter}{List of Abbreviations} 
\symbolsize=3.5cm% sirka sloupecku pro symboly, je mozno zmensit pokud jsou kratke
\vspace{7mm}
\begin{symboly}

\item[CNN] Convolutional Neural Network
\item[OS] Operating System
\item[ANN] Artificial Neural Network
\item[ReLU] Rectified Linear Unit
\item[MSE] Mean Squared Error
\item[SGD] Stochastic Gradient Descent
\item[RGB] Red-Green-Blue
\item[FCN] Fully Connected Network
\item[MCDO] Monte Carlo Dropout
\item[IoU] Intersection over Union
\item[CPU] Central Processing Unit
\item[GPU] Graphics Processing Unit
\item[AI] Artificial Intelligence
\item[BAIR] Berkeley AI Research
\item[CPW] Caffe Python Wrapper
\item[SSD] Solid-State Drive
\item[RAM] Random-Access Memory
\item[LTS] Long-Term Support

\end{symboly}
% nutné
\chapter*{List of Attachments}
\addcontentsline{toc}{chapter}{List of Attachments} 
\vspace{7mm}
\begin{enumerate}
\item Architecture of SegNet (all variants): \textit{scheme.pdf}

\end{enumerate}

% není povinné
\end{document}
