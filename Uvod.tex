\chapter{Úvod}
S postupným vývojem mikroelektroniky a řídících inteligentních systému se otevírá možnost implementovat složitější řídící algoritmy do zařízení a robotů o relativně malých rozměrech. Tato práce se zabývá návrhem a realizací nestandardního robota, který vykazuje značnou míru nestability při balancování na sférické základně. Jedná se o typicky mechatronickou úlohu, jelikož v sobě zahrnuje funkční celky z více technických oborů (konstrukční návrh, elektronika, inteligentní řízení). Tyto přitom neoddělitelně tvoří výsledný mechatronický produkt. Ballboti, jak zní dnes již vžitý název pro tyto druhy robotů, mají některé prvky chování společné s inverzním kyvadlem a jejich řídící algoritmus, zajišťující setrvání v nestabilní rovnováze, má velmi podobnou strukturu. Zde se však jedná o úlohu mající v klasické konstrukční variantě pět stupňů volnosti.

	Systémy tohoto druhu nejsou zatím v praxi moc rozšířené. Jejich rozvoj může však do budoucna umožnit využití v samostatně pohybujících se robotických celcích, u kterých je kladen důraz na obratnost v přelidněných prostorách, úzkých chodbách apod. Lze také očekávat zájem zábavního průmyslu o zpřístupnění robota široké veřejnosti.
	
	V rámci závěrečných studentských projektů bylo v mechatronické laboratoři MechLab na Fakultě strojního inženýrství VUT v Brně realizováno několik projektů s podobnými rysy, vycházejících z podobných fyzikálních předpokladů. Například práce Františka Zouhara \cite{zouhar}, zabývající se návrhem nestabilního balancujícího vozidla, uvádí v rešeršní části práce mj. ballbota jakožto zařízení podobného charakteru. V laboratoři dále vzniká řada provedení čistě inverzního kyvadla.

